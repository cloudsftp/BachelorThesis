\subsection{Problem Implementation}

For the implementation of UCPs for classical computers
the MINLP formulation is implemented using the Python library Pyomo.
Pyomo is open-source and supports the formulation of various optimization problems,
particulary mixed-integer nonlinear problems.
It also supports performing the optimization of the problems
with standard open-source or commercial solvers.
\cite{hart2011pyomo}

The program is given a UCP object, that holds information
about the power demands and the available power plants.
It then creates a ''ConcreteModel'' from this information.
This model of the MINLP formulation of the UCP
has three two-dimensional matrices as variables.
The first two are the variables fo the commitment $u_{i, t}$
and the power output $p_{i, t}$ for each power plants at each time instance.
These are the free variables of the model that will be adjusted so that the objective funtion is minimized.
The last matrix is for $S_{i, t}$, the startup and shutdown costs.
These are not free but depend on the data from the UCP object.

\todo{Include listing?}

\subsubsection{The Objective Function}

The formula (\ref{formula:minlp.obj}) is modeled using a Objective object provided by the Pyomo framework.
It depends on the model variables for $u_{i, t}$, $p_{i, t}$ and $S_{i, t}$.
The coefficients $A_i$, $B_i$ and $C_i$ are taken from the UCP object.

\todo{Include listing?}

\todo{startup shotdown costs}

\subsubsection{Constraints}

\todo{Pyomo model constraints}

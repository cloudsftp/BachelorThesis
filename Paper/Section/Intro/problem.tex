\section{The Unit Commitment Problem}

The Unit Commitment Problem (UCP) generally is concerned with the scheduling or ''commitment''
of thermal power plants over a given time period.
Where thermal power plants can be coal, oil, gas or nuclear power plants.
The commitment of the power plants should be scheduled in such a way,
that the power demand or ''load'' at each moment of the time period
is satisfied by the power output of the power plants.
While satisfying the load, the fuel cost should be minimal.
The fuel cost depends on the commitment of each power plant at each time ($u_{i, t}$),
the power output of each power plant at each time ($p_{i, t}$),
the power plant specific cost function coefficients ($A_i$, $B_i$ and $C_i$),
as well as start-up and shut-down costs ($A_{i}^{U}$ and $A_{i}^{D}$).
Other than the load, the problem also consideres various other constraints of the power plants.
These include the minimum and maximum power output of each plant ($P_{min,i}$ and $P_{max,i}$),
the ramping constraints of each plant,
and the minimum up- and down-time.
\cite{Baldick1995}

In this paper, only load serves as a constraint to the commitment of the thermal power plants.
\todo{Problem description, cost function formula}


\todo{Rationale for MINLP vs MILP}
The problem is often mathematically formulated as a Mixed-Integer Non-Linear Problem (MINLP).
MINLP is a class of optimization problems, with a non-linear objective funtion, integer and floating-point variables,
as well as various constraints on the variables.
These models are non-convex in the case of the UCP, due to the binary nature of the commitment variables $u_{i, t}$.
\cite{Baldick1995, Abujarad2017}
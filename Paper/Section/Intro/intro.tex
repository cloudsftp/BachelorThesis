\chapter{Introduction}

The scheduling of conventional thermal power plants is highly important
with the rise of renewable energies.
It is used to schedule hydro- and thermal power plants of a system.
\cite{Banos2011}
Also it can be used with wind energy and thermal power plants
to minimize the cost of incorrect weather predictions.
These approaches are also called risk-based
because they model the risk of the weather predictions being wrong.
\cite{Chen2008,Abujarad2017}

The number of deployed smart meters in the european union was 99 million in 2018.
A study by the European Commision projected the number to reach 123 million by 2020.
\cite{Vlachogiannis2019}
With the increasing number of of smart meters,
the amount of information about energy consumption is growing.
The data is available with a time resolution that was not yet possible.
There is a possibility, that distributed generation will be used in the future.
This means that for example the photo voltaic systems of consumers
give energy to the grid if it is not needed by the consumers.
Additionally electrical vehicles can be used as energy sources when plugged into a charger.
This gives additional degrees of freedom for the energy procurement.
\cite{Aiello2016,Zhang2016}

Due to the high time resolution in the data produced by the smart meters,
the prediction of future power consumption is possible on a house to house basis.
This allows for a much better overall prediction of the power consumption of towns or cities.
\cite{Basu2013}
It presents the possibility to schedule thermal power plants much more precisely
and with a higher time resolution.

\section{The Unit Commitment Problem} \label{intro:ucp}

The Unit Commitment Problem (UCP) generally is concerned with the scheduling or ''commitment''
of thermal power plants over a given time period.
Where thermal power plants can be coal, oil, gas or nuclear power plants.
The commitment of the power plants should be scheduled in such a way,
that the power demand or ''netwrok load'' at each moment of the time period
is satisfied by the power output of the power plants.
While satisfying the netwrok load, the fuel cost should be minimal.
The fuel cost depends on the commitment of each power plant at each time ($u_{i, t}$),
the power output of each power plant at each time ($p_{i, t}$),
the power plant specific cost function coefficients ($A_i$, $B_i$ and $C_i$),
as well as start-up and shut-down costs ($A_{i}^{U}$ and $A_{i}^{D}$).
Other than the netwrok load, the problem also consideres various other constraints of the power plants.
These include the minimum and maximum power output of each plant ($P_{min,i}$ and $P_{max,i}$),
the ramping constraints of each plant,
and the minimum up- and down-time.
\cite{Baldick1995}

In this paper, only the network load serves as a constraint to the commitment of the thermal power plants.
So it consists of the cost function $f_{i, t}$ that is to be minimized,
the network load constraint for each time $t$,
and the power output constraint for each unit $i$ and time $t$.

The UCP is non-linear in nature, because of the quadratic cost function $f_{i, t}$.
It also has floating-point variables as well as binary variables,
where binary variables can be represented as integer variables with domain $\{0, 1\}$.
The class of Mixed Integer Non-linear Problems (MINLP) is a class of optimization problems,
with a non-linear objective funtion,
integer and floating-point variables,
as well as various constraints on the variables.
This is the reason, that the UCPs are often formulated as MINLPs.
These models are non-convex in the case of the UCP, due to the binary nature of the commitment variables $u_{i, t}$.
\cite{Baldick1995, Abujarad2017}
\section{Quantum Computing}

An introduction to quantum computing.
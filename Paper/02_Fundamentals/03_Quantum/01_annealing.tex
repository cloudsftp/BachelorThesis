\subsection{Annealing-based Quantum Computing}
\label{fundamentals:annealing}

Annealing-based quantum computers are designed to tackle optimization problems well.
The quantum processing unit (QPU) takes an Ising-model as input and embeds the biases on itself.
For this, it uses couplers between its qubits.
Then quantum tunneling is exploited to find a qubit-state where the system's energy is minimal.
The result is the state of the Ising-model variables that minimize the model~\cite{Boixo2013}.

Ising models and QUBOs are equivalent, and the conversion is computationally inexpensive.
The difference is the domain of the variables~\cite{Bian2010}.
In Ising models, the domain is $\{-1, 1\}$, and in QUBOs, it is $\{0, 1\}$.
This work considers QUBOs rather than Ising models.
The formula (\ref{formula:qubo.form}) shows the structure of a QUBO.
$a$ are called linear biases because they depend on one variable, and $b$ are called quadratic biases because they depend on two variables.
$c$ is called the constant bias because it depends on no variables and is constant.
\begin{align}
  \label{formula:qubo.form}
  E(v) = & \quad
  \sum_i a_i \cdot v_i
  + \sum_{i < j} b_{i, j} \cdot v_i \cdot v_j
  + c
\end{align}

\subsubsection{Hybrid Quantum-annealing}

The number of qubits on a QPU is limited, and so are the couplings between them that the QPU uses to embed quadratic biases.
When a problem has too many variables or quadratic biases, it's impossible to find an embedding for the QPU~\cite{Bernoudy2020}.

A hybrid sampler is an algorithm that runs on a classical computer and a quantum computer.
The classical computer receives the problem and attempts to solve it.
It uses the quantum computer for subproblems that the quantum computer can solve faster~\cite{Zhang2016}.

D-Wave's Advantage system has more than $5, 000$ qubits with $15$-way connectivity~\cite{D-Wave2020}.
It can solve problems with up to $5, 000$ binary variables if the quadratic biases are $\leq 15$ for every variable.
If there are some variables with more quadratic biases, they can be distributed onto different qubits.
At some point, this is not possible anymore, and a hybrid algorithm must be used.
D-Wave's hybrid sampler can handle fully connected QUBOs of up to $20, 000$ variables.
It can handle non-fully connected QUBOs of up to $1, 000, 000$ variables, but the number of biases has to be smaller than $200, 000, 000$~\cite{Bernoudy2020}.

\subsubsection{Discrete Optimization}

D-Wave also has a hybrid sampler that can handle problems with discrete variables instead of binary as in QUBOs.
These problems have to be a Discrete Quadratic Model (DQM).
Formula (\ref{formula:dqm.form}) shows the form of such DQMs.
The sampler can handle a maximum of $5, 000$ variables with up to $10, 000$ values per variable.
The maximum number of biases the sampler supports is $2, 000, 000, 000$~\cite{DQMHybrid2020}.
\begin{align}
  \label{formula:dqm.form}
  E(v) =
  & \quad \sum_i a_i \cdot v_i + \sum_{i < j} b_{i, j} \cdot \left( v_i \otimes v_j \right) + c
\end{align}

Note that in the formula (\ref{formula:dqm.form}), $a_i$ is a vector that holds the linear biases for the variable $i$.
$a_i \cdot v_i$ is the scalar product of the linear biases with the value of variable $i$.
$b_{i, j}$, on the other hand, is a matrix that holds the quadratic biases of the variable $i$ and $j$.
$b_{i, j} \cdot \left( v_i \otimes v_j \right)$ is the Frobenius Inner Product $\langle b_{i, j}, v_i \otimes v_j \rangle_F$ that is defined as the sum of all elements of both matrices multiplied element-wise.

\section{Quantum Computing}
\label{fundamentals:quantum}

Quantum computers use quantum bits (qubits) instead of bits that classical computers use.
While a bit can be either in the state $0$ or $1$, a qubit can be in a so-called superposition of two states.
States of qubits are generally written using the Dirac-Notation, also known as the Bra-Ket-Notation.
The two states, the superposition is a combination of, are called the basis.
The most basic basis is the basis $\{ \ket{0}, \ket{1} \}$.
The formula (\ref{formula:quantum.superposition}) describes such a superposition.
When one measures the state of the qubit, it collapses into one of the base states, which is the measurement result.
The probability of either basis state being measured is the absolute square of the complex amplitude of the basic state in the superposition.
Formula (\ref{formula:quantum.superposition.probability}) shows this for the state $\ket{0}$~\cite{Vedral1998}.
\begin{subequations}
\begin{align}
  \ket{\phi} & = \alpha \cdot \ket{0} + \beta \cdot \ket{1}
  = \begin{pmatrix}
    \alpha \\ \beta
  \end{pmatrix}
  & \text{with } \alpha, \beta \in \mathbb{C}, \alpha^2 + \beta^2 = 1
  \label{formula:quantum.superposition}
  \\
  p_0 & = |\alpha|^2
  \label{formula:quantum.superposition.probability}
\end{align}
\end{subequations}

Multiple qubits chained together form a quantum register.
In such a register, multiple qubits might be entangled.
That means that the measurement of one qubit depends on the measurement of the other qubit.
The formula (\ref{formula:quantum.superposition.register}) describes a quantum register with $n$ qubits.
If one can't separate the formula into single qubits, the state is entangled.
The probability of measuring a specific state of the quantum register is given by the absolute square of the amplitude of that state
Formula (\ref{formula:quantum.superposition.register.probability}) shoes this for all states $\ket{x}$~\cite{Vedral1998}.
\begin{subequations}
\begin{align}
  \ket{\Phi} & = \sum_{x \in \{0, 1\}^n} \alpha_x \cdot \ket{x}
  = \begin{pmatrix}
    \alpha_{0 \ldots 00} \\ \alpha_{0 \ldots 01} \\ \alpha_{0 \ldots 10} \\ \ldots \\ \alpha_{1 \ldots 11}
  \end{pmatrix}
  & \text{with } \forall x \in \{0, 1\}^n: \alpha_x \in \mathbb{C},
  \sum_{x \in \{0, 1\}^n} \alpha_x^2 = 1
  \label{formula:quantum.superposition.register}
  \\
  p_x & = \alpha_x^2
  \label{formula:quantum.superposition.register.probability}
\end{align}
\end{subequations}

\subsection{Annealing-based Quantum Computing}
\label{backg:annealing}

Annealing-based quantum computers are designed to tackle optimization problems well.
The quantum processing unit takes an Ising as input and embeds the biases on itself.
For this, it uses couplers between its qubits.
Then quantum tunneling is exploited to find a state of the qubits where the energy of the QPU is minimal.
This is also the state of the Ising model variables that minimize the model.
\cite{Boixo2013}

Ising models and QUBOs are equivalent, and the conversion is computationally inexpensive.
The difference is the domain of the variables.
\cite{Bian2010}
In Ising models, the domain is $\{-1, 1\}$, and in QUBOs, it is $\{0, 1\}$.
This work considers QUBOs rather than Ising models.
The formula \ref{formula:qubo.form} shows the structure of a QUBO.
$a$ are called linear biases because they depend on one variable, and $b$ are called quadratic biases because they depend on two variables.

\begin{align}
  \label{formula:qubo.form}
  E(v) = & \quad
  \sum_i a_i \cdot v_i
  + \sum_{i < j} b_{i, j} \cdot v_i \cdot v_j
  + c
\end{align}

\subsection{Discrete Optimization}

\begin{subequations}
\begin{align}
  \label{formula:dqm.form}
  E(v) =
  & \quad \sum_i a_i v_i + \sum_{i < j} b_{i, j} v_i v_j + c \\
  = & \quad \sum_i a_i v_i + \sum_{i < j} b_{i, j} \left( v_i \otimes v_j \right) + c
\end{align}
\end{subequations}

\todo{Explain which what scalar multiplications are used}

\subsection{Gate-based Quantum Computing}

In the gate-based model of quantum computers, gates represent the manipulation of qubits.
The qubits and quantum gates form a quantum circuit.

Single qubit gates manipulate only one qubit at a time.
Double qubit gates manipulate two qubits and can entangle two qubits.
The formula (\ref{formula:gate.x}) shows an example of a single qubit gate.
This gate is called the Pauli-X gate, and it swaps the amplitudes of the two basis states $\ket{0}$ and $\ket{1}$.
The formula (\ref{formula:gate.cnot}) shows an example of a double qubit gate.
This gate is called the CNOT, and it applies the Pauli-X gate to the second qubit if the first qubit is in the base state $\ket{1}$.
The CNOT gate entangles the involved qubits.
\begin{subequations}
\begin{align}
  X & = \ket{0} \bra{1} + \ket{1} \bra{0}
  & = \begin{pmatrix}
    0 & 1 \\ 1 & 0
  \end{pmatrix}
  \label{formula:gate.x}
  \\
  H & = \frac{1}{\sqrt{2}} \left(
    \left( \ket{0} + \ket{1} \right) \bra{0}
    + \left( \ket{0} - \ket{1} \right) \bra{0}
  \right)
  & = \frac{1}{\sqrt{2}} \begin{pmatrix}
    1 & 1 \\ 1 & -1
  \end{pmatrix}
  \label{formula:gate.hadamard}
  \\
  CNOT & = \ket{00} \bra{00} + \ket{01} \bra{01} + \ket{10} \bra{11} + \ket{11} \bra{10}
  & = \begin{pmatrix}
    1 & 0 & 0 & 0 \\
    0 & 1 & 0 & 0 \\
    0 & 0 & 0 & 1 \\
    0 & 0 & 1 & 0
  \end{pmatrix}
  \label{formula:gate.cnot}
\end{align}
\end{subequations}

A quantum circuit consists of one or more quantum gates that are operating on one or more qubits.
The quantum circuit \ref{figure:gate.deutsch.circuit} depicts the Deutsch algorithm for the function $f: x \mapsto x$.
The dot with the line to the circled plus stands for the CNOT gate, and the H gates stand for Hadamard gates.
A Hadamard gate puts qubits that are in a basic state into a superposition where both outcomes of the measurement are equally likely.
It is described by the formula (\ref{formula:gate.hadamard}).
The last gate stands for a measurement.
In this case only the first qubit is measured.
\cite{Deutsch1985}
\begin{figure}[!h]
  \centering
  \includegraphics[width=0.5 \textwidth]{02_Background/deutsch_algorithm_circuit.png}
  \caption{Deutsch Algorithm for $f: x \mapsto x$}
  \label{figure:gate.deutsch.circuit}
\end{figure}

A quantum computer that implements this model of computing would be more powerful than any Turing machine.
It can simulate any finite physical system with polynomial time complexity, including systems with quantum effects.
The complexity class is called Bounded-error Quantum Polynomial-time (BQP).
There exists no known algorithm for Turing machines to accomplish this~\cite{Deutsch1985, Shor1998}.

\subsubsection{Grover Algorithm}

The Grover Algorithm is a quantum algorithm for unstructured search.
It can find an element that satisfies some condition $C$ in an unstructured list with the time complexity of $O(\sqrt{n})$~\cite{Grover1996}.
If multiple elements satisfy the condition, every execution of the algorithm yields only one of those elements~\cite{Boyer1998}.

The algorithm consists on 3 parts:
\begin{enumerate}
  \item Initialize the input register with the equal superposition $\ket{\Phi} = \frac{1}{\sqrt{N}} \sum_{x=0}^N \ket{x}$
  \item \label{fundamentals:quantum.grover.iteration}
  Repeat the following steps $O(\sqrt{N})$ times:
  \begin{enumerate}[label=\alph*)]
    \item Rotate all states that satisfy $C$ by $\pi$.
    \item Apply the diffusion operator $D$.
    It is defined as \begin{align}
      D_{i, j} = \begin{cases}
        \frac{2}{N} & i \neq j\\
        \frac{2}{N} - 1 & else
      \end{cases}
    \end{align}
  \end{enumerate}
  \item \label{fundamentals:quantum.grover.sampling}
  Sample the input register.
\end{enumerate}

Step \ref{fundamentals:quantum.grover.iteration} is called the Grover-iteration $G$.
Its first part that rotates desired states is called an oracle.
The Grover-iteration amplifies the amplitudes of desired states.
This amplification increases the probability of measuring those states in step \ref{fundamentals:quantum.grover.sampling}~\cite{Grover1996}.

\subsubsection{Optimization}
\label{fundamentals:quantum.grover.optimization}

Optimizing QUBOs on gate-based quantum computers is possible by iteratively searching for better values of the QUBO variables.
The algorithm iteratively uses the Grover search to search for better values for the QUBO variables.
It uses two quantum registers to represent the variables of the QUBO ($\ket{x}$), and the energy value of the QUBO ($\ket{z}$)~\cite{Gilliam2019}.

The oracle consists of multiple rotations of qubits of $\ket{z}$.
The qubits of $\ket{x}$ control the rotations.
The rotations add the biases of the QUBO to the value of the quantum register in the Fourier-domain.
After that, the oracle subtracts the current minimum energy value $y$ from the value of $\ket{z}$.
The inverse Fourier-transformation converts the $\ket{z}$ register to a Two's Complement basis.
The most significant qubit of $\ket{z}$ signals the sign of the value and controls the typical oracle used to rotate the desired states~\cite{Gilliam2019}.

This way, the measurement of $\ket{x}$ yields values for the QUBO variables that correspond to a lower energy value $y'$ than the minimum found before $y$.
In the next iteration, the oracle subtracts $y'$ from $\ket{z}$ after adding the biases.
By repeating this process, the algorithm identifies the values for the QUBO variables that minimize the energy~\cite{Gilliam2019}.


\subsection{State of the art}

\subsubsection{Annealing-based Quantum Computers}

D-Wave is the leading manufacturer of annealing-based quantum computers.
Their newest model, ``Advantage'', has over $5, 000$ qubits and over $35, 000$ couplers.
The connectors connect the qubits in a $P_{16}$-Graph, also called ``Pegasus''.
This graph has the degree $16$, which means that every qubit has a connection with $15$ other qubits.
D-Wave released this model in 2020~\cite{D-Wave2020, Zbinden2020}.

The model before this, ``2000Q'', has over $2, 000$ qubits and over $6, 000$ couplers.
The connectors connect the qubits in a $C_{16}$-Graph, also called ``Chimera''.
This graph has the degree $6$, which means that every qubit has a connection with $5$ other qubits.
D-wave released that model in 2017~\cite{D-Wave2020, Zbinden2020}.

D-wave, therefore, increased their qubit number significantly in the last years.
The qubit count increased by $150\%$ in just $3$ years.
The coupler count increased by $580\%$ in the same time frame, although this number is misleading.
The more meaningful metric anout the coupler count is the qubit connectivity, the so-called ``degree'', which increased by $133\%$.

\begin{figure}[!ht]
  \centering
  \includegraphics[width=0.7 \textwidth]{02_Background/dwave_qbits_history.png}
  \caption{Number of Qubits in D-wave QPUs over the Years \cite{D-Wave2018, D-Wave2020}}
  \label{figure:annealing.processors.history}
\end{figure}

Figure \ref{figure:annealing.processors.history} shows the increase of the number of qubits that are in D-Wave's QPUs.
The number of qubits doubles all $2$ years.

\subsubsection{Gate-based Quantum Computers}

IBM is the leading manufacturer of Gate-based Quantum Computers.
Their newest model, ``Hummingbird'', has exactly $65$ qubits.
It debuted in 2020.
Unfortunately, information about the performance of this processor is not available to the public~\cite{IBMRoadmap2020}.

The model before that, ``Falcon'', has exactly $27$ qubits.
It debuted in 2019.
IBM was able to achieve a quantum volume of $64$ on this processor~\cite{IBMRoadmap2020}.

Quantum volume is a metric that characterizes the power of quantum processors.
It depends on the number of qubits and the number of gates that a quantum circuit can have without producing significant errors~\cite{Bishop2017}.
IBM managed to achieve a $100\%$ increase of the quantum volume metric every year since 2017~\cite{IBMqv2020}.
This is illustrated in figure \ref{figure:gate.qv.history}.

Even though the quantum volume is an improved measurement of the power of a quantum processor, the number of qubits is still important.
It determines how big the problems can be for the quantum processor to solve them.
IBM promises at least a doubling of the qubit count every year in their 2020 roadmap.
It projects its 2023 quantum processors to have $1, 121$ qubits.
Figure \ref{figure:gate.qbits.projection} ilustrates the projected qubit count~\cite{IBMRoadmap2020}.

\begin{figure}[!ht]
  \begin{subfigure}[b]{0.5 \textwidth}
    \centering
    \includegraphics[width=\textwidth]{02_Background/ibm_qv_history.png}
    \caption{Quantum Volume achieved\\ of IBM Quantum Processors \cite{IBMqv2020}}
    \label{figure:gate.qv.history}
  \end{subfigure}
  \begin{subfigure}[b]{0.5 \textwidth}
    \centering
    \includegraphics[width=\textwidth]{02_Background/ibm_qubits_projection.png}
    \caption{Projected Qubit Count\\ of IBM Quantum Processors \cite{IBMRoadmap2020}}
    \label{figure:gate.qbits.projection}
  \end{subfigure}
  \caption{Characteristics of IBM Quantum Processors}
\end{figure}

\section{Mixed-Integer Non-linear Problems}
\label{fundamentals:minlp}

Mixed-integer Non-linear Problems (MINLPs) are a class of optimization problems that fit a specific form.
The form of the problems is displayed in formula \ref{formula:minlp.form}.

\begin{subequations}
  \begin{align}
    \min_x \quad
    & f(x)
    \label{formula:minlp.form.objective}
    \\
    s.t. \quad
    & g_j(x) \leq 0
    & \forall j \in M
    \label{formula:minlp.form.constraints}
    \\
    & x_i^l \leq x_i \leq x_i^u
    & \forall i \in N_0
    \label{formula:minlp.form.limits}
    \\
    & x_i \in \mathbb{Z}
    & \forall i \in N_0^I \subseteq N_0
    \label{formula:minlp.form.integer}
  \end{align}
  \label{formula:minlp.form}
\end{subequations}

where $f(x)$ and all $g(x)$ may be non-linear.
That is why the class of problems is called non-linear.

The first line \ref{formula:minlp.form.objective} is the objective function.
The goal is to find an input $x \in \mathbb{R}^n$ that minimizes $f(x)$ while also respecting the constraints discussed in the next paragraph.

The second line \ref{formula:minlp.form.constraints} lists all the constraints that involve multiple parts of the input x.
The input $x$ must not violate any of the constraints.
Line \ref{formula:minlp.form.limits} sets limits for the elements $x_i$ of $x$.
And the last line \ref{formula:minlp.form.integer} states that some of the elements $x_i$ are integers rather than rational numbers.
That is why the class of problems is called mixed-integer.
\cite{Belotti2009}

Optimizing MINLPs is NP-complete.
\cite{Bienstock1996}
That means that no known algorithm can optimize any MINLP with polynomial time complexity.

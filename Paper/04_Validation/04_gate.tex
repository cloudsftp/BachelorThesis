\section{Performance of Gate-based Quantum Computers}

The implementation of the QUBO as described in section \ref{approach:gate.implement} of the UCP builds a quantum circuit for the first iteration of the optimization algorithm described in section \ref{backg:quantum.grover.optimization}.
Unfortunately, this quantum circuit is too big to be executed on any of IBM's public quantum computers for the smallest UCP that this work considers.

The number of qubits needed just for the QUBO variables is given by the formula (\ref{formula:qubo.num.qubits}).
This is considered the first register of the quantum computer.
In the case of the smallest UCP this work considers with the $4$ power plants listed in section \ref{table:validation.data.plants} and $2$ time instances, the number of qubits would be $\left( 2^6 + 2^6 + 2^7 + 2^8 \right) \cdot 2 = 1024$.
With every additional time instance of the UCP, another $1024$ qubits are needed.

Additionally to these qubits, the quantum circuit needs a second register for storing the result of the QUBO dependent on the QUBO variables of the first register.
This number also grows with the QUBO size.

This means that the gate-based quantum computers right now do not have enough qubits to handle the problems this work considers.

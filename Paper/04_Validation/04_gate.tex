\section{Performance of Gate-based Quantum Computers}

The implementation of the QUBO as described in section \ref{approach:gate.implement} of the UCP builds a quantum circuit for the first iteration of the optimization algorithm described in section \ref{backg:quantum.grover.optimization}.
Unfortunately, this quantum circuit is too big to be executed on any of IBM's public quantum computers for the smallest UCP that this work considers.

The number of qubits needed just for the QUBO variables is the same as the number of binary variables in the QUBO.
The number of variables in a QUBO is given by the formula (\ref{formula:qubo.num.variables}).
These qubits are considered the first quantum register of the quantum circuit.
In the case of the smallest UCP, this work considers with the $4$ power plants listed in section \ref{table:validation.data.plants} and $2$ time instances, the number of qubits would be $1, 024$.
With every additional $2$ time instances of the UCP, the algorithm needs another $1, 024$ qubits.

Additionally to this quantum register, the algorithm needs a second quantum register for storing the result of the QUBO dependent on the QUBO variables of the first register.
The domain of values the QUBO produces grows with the size of the QUBO.
It is proportional to the number of biases assuming the average value of the biases stays the same.

Regardless of the number of gates the quantum circuit would have, the gate-based quantum computers right now do not have enough qubits to handle the problems this work considers.

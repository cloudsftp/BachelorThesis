\section{Related Work}

\subsection{Unit Commitment Problem}

Many methods for optimizing the UCP described in section \ref{backg:ucp} exist in the literature.
Abujarad et al. \cite{Abujarad2017} listed and compared those methods.
They listed the use of Mixed-integer Linear Programming as accurate but exponentially expensive.
All other methods give a sub-optimal solution.
\cite{Abujarad2017}

When dealing with cost functions of thermal power plants, they seldom are linear.
Most commonly, they are quadratic.
Baldick \cite{Baldick1995} generalized the formulation of the UCP as Mixed-integer Non-linear Problems, where he assumed a quadratic cost function.

\subsection{Quantum Computing for Unit Commitment}

Ajagekar and You \cite{Ajagekar2019} explore the utility of quantum computing for optimization problems in the energy domain.
They also considered the UCP.
They formulated a UCP as a QUBO and used D-Wave quantum hardware (the D-Wave 2000Q, to be precise) to optimize that model of the UCP.
Compared to the optimization on a classical computer using the Gurobi solver, the quantum option did perform terribly.
\cite{Ajagekar2019}

\subsection{Discrete Quadratic Model Optimization on Quantum Computers}

At the end of 2020, D-Wave released a hybrid solver that can solve Discrete Quadratic Models.
\cite{DQMHybrid2020}
The DQM model has a large advantage over the QUBO model when modeling problems with discrete or continuous variables.
The QUBO can only have binary variables.
Modeling discrete variables takes many binary variables and many more quadratic biases to ensure only one binary variable corresponding to one discrete variable is $1$.
This phenomenon also appears in the QUBO formulation of the UCP by Agagekar and You.
\cite{Ajagekar2019}

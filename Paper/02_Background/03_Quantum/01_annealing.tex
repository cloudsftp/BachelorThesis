\subsection{Annealing-based Quantum Computing}
\label{backg:annealing}

Annealing-based quantum computers are designed to tackle optimization problems well.
The quantum processing unit takes an Ising as input and embeds the biases on itself.
For this, it uses couplers between its qubits.
Then quantum tunneling is exploited to find a state of the qubits where the energy of the QPU is minimal.
This is also the state of the Ising model variables that minimize the model.
\cite{Boixo2013}

Ising models and QUBOs are equivalent, and the conversion is computationally inexpensive.
The difference is the domain of the variables.
\cite{Bian2010}
In Ising models, the domain is $\{-1, 1\}$, and in QUBOs, it is $\{0, 1\}$.
This work considers QUBOs rather than Ising models.
The formula \ref{formula:qubo.form} shows the structure of a QUBO.
$a$ are called linear biases because they depend on one variable, and $b$ are called quadratic biases because they depend on two variables.

\begin{align}
  \label{formula:qubo.form}
  E(v) = & \quad
  \sum_i a_i \cdot v_i
  + \sum_{i < j} b_{i, j} \cdot v_i \cdot v_j
  + c
\end{align}

\subsection{Discrete Optimization}

\begin{subequations}
\begin{align}
  \label{formula:dqm.form}
  E(v) =
  & \quad \sum_i a_i v_i + \sum_{i < j} b_{i, j} v_i v_j + c \\
  = & \quad \sum_i a_i v_i + \sum_{i < j} b_{i, j} \left( v_i \otimes v_j \right) + c
\end{align}
\end{subequations}

\todo{Explain which what scalar multiplications are used}

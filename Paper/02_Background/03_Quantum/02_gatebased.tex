\subsection{Gate-based Quantum Computing}

In the gate-based model of quantum computers, gates represent the manipulation of qubits.
The qubits and quantum gates form a quantum circuit.

Single qubit gates manipulate only one qubit at a time.
Double qubit gates manipulate two qubits and can entangle two qubits.
The formula (\ref{formula:gate.x}) shows an example of a single qubit gate.
It swaps the amplitudes of the two basis states $\ket{0}$ and $\ket{1}$.
The formula (\ref{formula:gate.cnot}) shows an example of a double qubit gate.
It applies the $X$-gate to the second qubit if the first qubit is in the base state $\ket{1}$.
This entangles both qubits with each other.
\begin{subequations}
\begin{align}
  \label{formula:gate.x}
  X & = \ket{0} \bra{1} + \ket{1} \bra{0}
  & = \begin{pmatrix}
    0 & 1 \\ 1 & 0
  \end{pmatrix}
  \\
  \label{formula:gate.hadamard}
  H & = \frac{1}{\sqrt{2}} \left(
    \left( \ket{0} + \ket{1} \right) \bra{0}
    + \left( \ket{0} - \ket{1} \right) \bra{0}
  \right)
  & = \frac{1}{\sqrt{2}} \begin{pmatrix}
    1 & 1 \\ 1 & -1
  \end{pmatrix}
  \\
  \label{formula:gate.cnot}
  CNOT & = \ket{00} \bra{00} + \ket{01} \bra{01} + \ket{10} \bra{11} + \ket{11} \bra{10}
  & = \begin{pmatrix}
    1 & 0 & 0 & 0 \\
    0 & 1 & 0 & 0 \\
    0 & 0 & 0 & 1 \\
    0 & 0 & 1 & 0
  \end{pmatrix}
\end{align}
\end{subequations}

A quantum circuit consists of one or more quantum gates that are operating on one or more qubits.
The quantum circuit \ref{figure:gate.deutsch.circuit} depicts the Deutsch algorithm for the function $f: x \mapsto x$.
The dot with the line to the circeled plus stands for the $CNOT$ gate and the $H$ gates stand for hadamard gates.
A Hadamard gate puts qubits that are in a basis state into a superposition where both outcomes of the measurement are equally likely.
It is described by the formula (\ref{formula:gate.hadamard}).
The last gate stands for a measurement.
In this case only the first qubit gets measured.
\cite{Deutsch1985}
\begin{figure}[!h]
  \centering
  \includegraphics[width=0.5 \textwidth]{02_Background/deutsch_algorithm_circuit.png}
  \caption{Deutsch Algorithm for $f: x \mapsto x$}
  \label{figure:gate.deutsch.circuit}
\end{figure}

\todo{Explain gate model}

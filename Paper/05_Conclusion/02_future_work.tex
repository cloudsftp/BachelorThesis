\section{Future Work}

\subsection{Unit Commitment Problem as Discrete Quadratic Model}

The conversion of the MINLP formulation of the UCP to a DQM formulation is not optimal.
While converting the UCP to a DQM, the proposed approach loses the ability to find the actual optimal solution to the UCP.
This is shown experimentally in section \ref{validation:comparison}.

There might be a better way to formulate a DQM for the UCP.
If there is a way that produces DQMs of the same size, it can achieve the same computational advantage as demonstrated in section \ref{validation:comparison} while finding the actual optimal solution.
The size of the DQM depends on the number of variables, their possible values, and the number of biases.

\subsection{Using Near-Optimal Result as Input for Classical Optimization}

The classical algorithm for optimizing MINLPs this work considers, ''Couenne'', can have a starting point for the optimization.
The result generated by the hybrid annealing-based DQM sampler could be used as this input.
The idea is to use the hybrid DQM sampler for the global optimization of starting and shutting down power plants and using the classical algorithm for the details of how much energy every plant should generate exactly.

\subsection{Hybrid Mixed-integer Non-linear Problem Solver}

Another promising approach is to find a hybrid algorithm for optimizing MINLP problems directly.
Then the reformulation as a DQM would not be necessary while achieving a computational advantage over purely classical algorithms.
\citeauthor{Ajagekar2020} demonstrated hybrid algorithms for similar mathematical optimization problems like the Mixed-integer Linear Problem (MILP).
\cite{Ajagekar2020}

Such a general-purpose hybrid MINLP solver would help not only the energy industry but also many other industries.
Most real-world industry problems can be formulated as an MINLP.
\cite{Belotti2009}
Thus most real-world industry problems could be solved using such a hybrid MINLP solver.

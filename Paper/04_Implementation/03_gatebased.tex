\section{Implementation for Gate-based Quantum Computers}
\label{implement:gate.qubo}

The implementation for the gate-based quantum computers uses the Qiskit framework.
Qiskit is an open-source framework for quantum computing on gate-based quantum computers.
IBM Research founded Qiskit~\cite{QiskitWeb, QiskitGitHub}.
After the program instantiates the QUBO using the \texttt{QuadraticProgram} class of the framework, it uses the class \texttt{GroverOptimizer} to optimize the QUBO.
This class handles executing the Grover Optimization algorithm that section \ref{fundamentals:quantum.grover.optimization} describes.
That includes building the quantum circuits of Grover algorithm instances for every iteration and the classical computation in between iterations.

This section omits the code snippets because they are the same as in section \ref{implementation:annealing.qubo}.
The only difference is how the program adds the biases to the model because it uses a different framework.

\subsubsection{Variables}

In this implementation, the variables are not implicit as with the UQO-client in section \ref{implementation:annealing.qubo.variables}.
The program instantiates all binary variables of the QUBO at the beginning of building the QUBO instance.
It assigns a name to every variable.
That name holds the indices of the corresponding power plant $i$, time $t$, and power output level $k$.

\subsubsection{Quadratic Biases}

The program adds the same quadratic biases as the implementation for annealing-based hardware that is described in section \ref{implementation:annealing.qubo.quadratic}.
It stores the biases in a Python \texttt{dict} with keys that are pairs of strings.
The strings are the names of the variables.

\subsubsection{Linear Biases}

The program adds the same linear biases as the implementation for annealing-based hardware that is described in section \ref{implementation:annealing.qubo.linear}.
It stores the biases in a Python \texttt{dict} with keys that are strings.
The strings are the names of the variables.

\subsubsection{Canstant Bias}

As mentioned before in section \ref{implementation:annealing.qubo.constant}, the constant bias does not change the input that produces an optimum of the QUBO.
But with this framework, the interface allows for a constant bias to be added to the QUBO.
The implementation computes the bias once, as specified in the formula (\ref{formula:qubo.result.constant}, and adds it to the QUBO instance.

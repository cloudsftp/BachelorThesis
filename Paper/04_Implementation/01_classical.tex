\section{Implementation on Classical Computers}

For the implementation of the MINLP for classical computers, this work uses the Python library Pyomo to build the MINLPs from the UCPs.
Pyomo is open-source and supports the formulation of various optimization problems, particularly mixed-integer nonlinear problems.
It also supports performing the optimization of the problems with standard open-source or commercial solvers~\cite{hart2011pyomo}.

The program is given a UCP object that holds information about the power demands and the available power plants.
It then creates a ``ConcreteModel'' from this information.
This model of the MINLP formulation of the UCP has three $\aI \times \aT$ matrices as variables.
The first two are the variables for the commitment $u_{i, t}$ and the power output $p_{i, t}$ for every power plant at every time.
These are the free variables of the model that will be adjusted so that the objective function is minimized.
The last matrix is for $s_{i, t}$, the startup and shutdown costs.
These are not free but depend on the data of the UCP object.

\subsection{The Objective Function}

The objective function (\ref{formula:minlp.obj}) is modeled using an \texttt{Objective} object provided by the Pyomo framework.
It depends on the model variables for $u_{i, t}$, $p_{i, t}$ and $s_{i, t}$.
Where $u_{i, t}$ and $p_{i, t}$ are free variables and $s_{i, t}$ are defined through the model.
The coefficients $A_i$, $B_i$, and $C_i$ are taken from the UCP object.
Listing \ref{implementation:classical.objective.code} shows the code that implements the objective function based on the free variables and the variables $s_{i, t}$ here called \texttt{startup\_shutdown\_cost}.

\begin{python}[
  float,
  caption={Implementation of the Objective Function for MINLPs},
  label={implementation:classical.objective.code}
]
def build_objective(self) -> None:
  '''
  builds the objective function of the MINLP
  '''
  def objective_function(model: ConcreteModel) -> Expression:
    plants: List[CombustionPlant] = self.ucp.plants

    return sum( # for every plant i
      sum( # and every time t
        model.u[(i, t)] * (
          plants[i].A +
          plants[i].B * model.p[i, t] +
          plants[i].C * model.p[i, t] ** 2
        ) + (
          model.startup_shutdown_cost[i, t]
        ) for t in model.T
      ) for i in model.I
    )

  self.model.o = Objective(rule=objective_function)
\end{python}

The startup and shutdown costs are defined through a method called ``disjunctive programming''.
This method makes it possible to formulate ``either-or'' conditions and set values of variables based on the value of other variables~\cite{Balas1983}.

With this method the value of $s_{i, j}$ set to either $A^U_i$, $A^D_i$ or $0$ depending on the values of $u_{i, t}$ and $u_{i, t-1}$.
This relation is defined mathematically by the constraint (\ref{formula:minlp.updowncost}).
And it is modeled by giving the model three disjunct events for each plant $i$ and time instance $t$:
\begin{align}
\begin{split}
  & \left( u_{i, t} < u_{i, t-1} \land s_{i, t} = A^U_i \right) \\
  \lor & \left( u_{i, t} > u_{i, t-1} \land s_{i, t} = A^D_i \right) \\
  \lor & \left( u_{i, t} = u_{i, t-1} \land s_{i, t} = 0 \right)
\end{split}
\end{align}

If $t = 0$, $u_{i, t-1}$ is defined as the initial commitment state of the power plant $i$.
The first part of the events is the condition and the second part defines the value of $s_{i, t}$.
Since the conditions are mutually exclusive, only one of the events will be true.
That means that $s_{i, t}$ has the value specified on the right side of the true condition.

\subsection{Constraints}

The first constraint is the load constraint.
It is specified by the formula (\ref{formula:minlp.load}).
It makes sure that the power demand is met by all the power plants $i$ at every time $t$.
The program creates a constraint list of length $\aT$.
Each element of the list makes sure that the sum of all power outputs at its corresponding time $t$ is equal to the power demand at that time instance.
Listing \ref{implementation:classical.load.constraint.code} shows the code that implements this constraint.

\begin{python}[
  float,
  caption={Implementation of the Load Constraint for MINLPs},
  label={implementation:classical.load.constraint.code}
]
def build_load_constraints(self) -> None:
  '''
  builds the constraints for the MINLP that make sure the power plants produce enough energy
  '''
  def load_constraint_rule(model: ConcreteModel, t: int) -> Expression:
    '''
    defines the rule for the constraints

    :model: MINLP
    :t: time index
    '''
    return self.ucp.loads[t] == sum(model.u[(i, t)] * model.p[(i, t)] for i in model.I)

  self.model.l_constr = Constraint(self.model.T, rule=load_constraint_rule)
\end{python}

The second constraint is the power level constraint.
It is specified by the formula (\ref{formula:minlp.power}).
It makes sure that the power output of every power plant $i$ is within its limits at every time $t$.
The program creates a constraint matrix of dimensions $\aT \times \aI$.
Each element of the constraint matrix makes sure the power output of the corresponding power plant $i$ at the corresponding time $t$ is within the limits of the power plant's constraints.
Listing \ref{implementation:classical.power.constraint.code} shows the code that implements this constraint.

\begin{python}[
  float,
  caption={Implementation of the Power Constraint for MINLPs},
  label={implementation:classical.power.constraint.code}
]
def build_power_constraints(self) -> None:
  '''
  builds the constraints for the MINLP that make sure every power output is inside the limits of the power plants
  '''
  def power_constraint_rule(model: ConcreteModel, i: int, t: int) -> Expression:
    '''
    defines the rule for the constraints

    :model: MINLP
    :i: plant index
    :t: time index
    '''
    return inequality(self.ucp.plants[i].Pmin, model.p[(i, t)], self.ucp.plants[i].Pmax)

  self.model.p_constr = Constraint(self.model.I, self.model.T, rule=power_constraint_rule)
\end{python}

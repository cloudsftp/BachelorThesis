\section{Problem}

The scheduling of conventional thermal power plants is highly important
with the rise of renewable energies.
It is used to schedule hydro- and thermal power plants of a system.
\cite{Banos2011}
Also it can be used with wind energy and thermal power plants
to minimize the cost of incorrect weather predictions.
These approaches are also called risk-based
because they model the risk of the weather predictions being wrong.
\cite{Chen2008,Abujarad2017}

The number of deployed smart meters in the european union was 99 million in 2018.
A study by the European Commision projected the number to reach 123 million by 2020.
\cite{Vlachogiannis2019}
With the increasing number of of smart meters,
the amount of information about energy consumption is growing.
The data is available with a time resolution that was not yet possible.
There is a possibility, that distributed generation will be used in the future.
This means that for example the photo voltaic systems of consumers
give energy to the grid if it is not needed by the consumers.
Additionally electrical vehicles can be used as energy sources when plugged into a charger.
This gives additional degrees of freedom for the energy procurement.
\cite{Aiello2016,Zhang2016}

Due to the high time resolution in the data produced by the smart meters,
the prediction of future power consumption is possible on a house to house basis.
This allows for a much better overall prediction of the power consumption of towns or cities.
\cite{Basu2013}
It presents the possibility to schedule thermal power plants much more precisely
and with a higher time resolution.

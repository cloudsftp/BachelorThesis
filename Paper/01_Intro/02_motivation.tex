\section{Motivation}

Optimizing the UCP takes a lot of time for large inputs.
Large inputs here means one of two things or both:
\begin{enumerate*}[label=(\roman*)]
  \item More units, i.e., power plants or
  \item A longer timeframe over which to schedule the units.
\end{enumerate*}
UCPs are optimization problems and thus, most of the time formulated as Mixed-integer Non-linear Problems.
\cite{Baldick1995}
What these are exactly is discussed in section \ref{backg:minlp}.
Important for now is that these optimization problems are NP-complete.
\cite{Li2005, Bienstock1996}

Quantum computing promises optimization algorithms with smaller time complexity.
\cite{Portnov2000, Ahuja1999, Gilliam2019, Ajagekar2020, Shaydulin2019}
There are two types of quantum processors.
Both types and their differences are introduced in section \ref{backg:quantum}.
One of which is designed for optimizations and the other which is a more general quantum processor.

Later it will become clear that most optimization algorithms that run solely on quantum computers will not be able to tackle the problem.
Today's quantum computers are described as Noisy Intermediate-Scale Quantum (NISQ) technology.
This is because the hardware is sensitive to noise and thus not able to run long computations.
Also, there are not many qubits in the latest quantum computers (intermediate-scale).
\cite{Leymann2020}
The solution is hybrid algorithms.
Such algorithms solve the problem at hand with classical computers and use quantum computers to speed up specific parts that are much faster on a quantum computer and realizable on todays NISQ hardware.
\cite{Ajagekar2020, Shaydulin2019}

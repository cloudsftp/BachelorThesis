% !TeX spellcheck = en-US
% !TeX encoding = utf8
% !TeX program = pdflatex
% !BIB program = biber
% -*- coding:utf-8 mod:LaTeX -*-


% vv  scroll down to line 200 for content  vv


\let\ifdeutsch\iffalse
\let\ifenglisch\iftrue
\input{90_Config/pre-documentclass}
\documentclass[
  % fontsize=11pt is the standard
  a4paper,  % Standard format - only KOMAScript uses paper=a4 - https://tex.stackexchange.com/a/61044/9075
  twoside,  % we are optimizing for both screen and two-side printing. So the page numbers will jump, but the content is configured to stay in the middle (by using the geometry package)
  bibliography=totoc,
  %               idxtotoc,   %Index ins Inhaltsverzeichnis
  %               liststotoc, %List of X ins Inhaltsverzeichnis, mit liststotocnumbered werden die Abbildungsverzeichnisse nummeriert
  headsepline,
  cleardoublepage=empty,
  parskip=half,
  %               draft    % um zu sehen, wo noch nachgebessert werden muss - wichtig, da Bindungskorrektur mit drin
  draft=false
]{scrbook}
\input{90_Config/config}


\usepackage[
  title={Quantum Computing for Smart Energy Optimizations},
  author={Fabian Weik},
  type=bachelor,
  institute=iaas, % or other institute names - or just a plain string using {Demo\\Demo...}
  course={Informatik},
  examiner={Prof.\ Dr.\ Marco Aiello},
  supervisor={Daniel Vietz, M. Sc.},
  startdate={October 12, 2020},
  enddate={April 12, 2020}
]{scientific-thesis-cover}

\usepackage{enumitem}

\newcommand{\bra}[1]{\langle #1 |}
\newcommand{\ket}[1]{| #1 \rangle}
\newcommand{\bracket}[2]{\langle #1 | #2 \rangle}

\input{90_Config/acronyms}

\makeindex

\begin{document}

%tex4ht-Konvertierung verschönern
\iftex4ht
  % tell tex4ht to create picures also for formulas starting with '$'
  % WARNING: a tex4ht run now takes forever!
  \Configure{$}{\PicMath}{\EndPicMath}{}
  %$ % <- syntax highlighting fix for emacs
  \Css{body {text-align:justify;}}

  %conversion of .pdf to .png
  \Configure{graphics*}
  {pdf}
  {\Needs{"convert \csname Gin@base\endcsname.pdf
      \csname Gin@base\endcsname.png"}%
    \Picture[pict]{\csname Gin@base\endcsname.png}%
  }
\fi

%\VerbatimFootnotes %verbatim text in Fußnoten erlauben. Geht normalerweise nicht.

\input{90_Config/commands}
\pagenumbering{arabic}
\Titelblatt

%Eigener Seitenstil fuer die Kurzfassung und das Inhaltsverzeichnis
\deftripstyle{preamble}{}{}{}{}{}{\pagemark}
%Doku zu deftripstyle: scrguide.pdf
\pagestyle{preamble}
\renewcommand*{\chapterpagestyle}{preamble}

\input{00_Preface/00_zusammenfassung}
\input{00_Preface/01_abstract}

\cleardoublepage


% BEGIN: Verzeichnisse

\iftex4ht
\else
  \microtypesetup{protrusion=false}
\fi

%%%
% Literaturverzeichnis ins TOC mit aufnehmen, aber nur wenn nichts anderes mehr hilft!
% \addcontentsline{toc}{chapter}{Literaturverzeichnis}
%
% oder zB
%\addcontentsline{toc}{section}{Abkürzungsverzeichnis}
%
%%%

%Produce table of contents
%
%In case you have trouble with headings reaching into the page numbers, enable the following three lines.
%Hint by http://golatex.de/inhaltsverzeichnis-schreibt-ueber-rand-t3106.html
%
%\makeatletter
%\renewcommand{\@pnumwidth}{2em}
%\makeatother
%
\tableofcontents

% Bei einem ungünstigen Seitenumbruch im Inhaltsverzeichnis, kann dieser mit
% \addtocontents{toc}{\protect\newpage}
% an der passenden Stelle im Fließtext erzwungen werden.

\listoffigures
\listoftables

%Wird nur bei Verwendung von der lstlisting-Umgebung mit dem "caption"-Parameter benoetigt
\lstlistoflistings
%ansonsten:
%\ifdeutsch
%  \listof{Listing}{Verzeichnis der Listings}
%\else
%  \listof{Listing}{List of Listings}
%\fi

%mittels \newfloat wurde die Algorithmus-Gleitumgebung definiert.
%Mit folgendem Befehl werden alle floats dieses Typs ausgegeben
%\ifdeutsch
%  \listof{Algorithmus}{Verzeichnis der Algorithmen}
%\else
%  \listof{Algorithmus}{List of Algorithms}
%\fi
%\listofalgorithms %Ist nur für Algorithmen, die mittels \begin{algorithm} umschlossen werden, nötig

% Abkürzungsverzeichnis
\printnoidxglossaries

\iftex4ht
\else
  %Optischen Randausgleich und Grauwertkorrektur wieder aktivieren
  \microtypesetup{protrusion=true}
\fi

% END: Verzeichnisse


% Headline and footline
\renewcommand*{\chapterpagestyle}{scrplain}
\pagestyle{scrheadings}
\pagestyle{scrheadings}
\ihead[]{}
\chead[]{}
\ohead[]{\headmark}
\cfoot[]{}
\ofoot[\usekomafont{pagenumber}\thepage]{\usekomafont{pagenumber}\thepage}
\ifoot[]{}

% Main content


\chapter*{Nomenclature}

\begin{tabular}{ll}
  $i \in \I$ & Indices of all units $\{0, 1, \ldots, I\}$ \\
  $t \in \T$ & Indices of all time steps $\{0, 1, \ldots, T\}$ \\
  $u_{i, t}$ & Commitment of unit $i$ at time $t$ \\
  $u_{i, -1}$ & Initial commitment of unit $i$ \\
  $p_{i, t}$ & Power output of unit $i$ at time $t$ \\
  $f_{i, t}$ & Cost function of unit $i$ at time $t$ \\
  $A_i$ & Constant cost function coefficient of unit $i$ \\
  $B_i$ & Linear cost function coefficient of unit $i$ \\
  $C_i$ & Quadratic cost function coefficient of unit $i$ \\
  $A^U_i$ & Start-up cost of unit $i$ \\
  $A^D_i$ & Shut-down cost of unit $i$ \\
  $s_{i, t}$ & Actual start-up or shut-down cost of unit $i$ at time $t$ \\
  $P_{min, i}$ & Minimum power output of unit $i$ \\
  $P_{max, i}$ & Maximum power ourput of unit $i$ \\
  $l_t$ & Power demand (``load'') at time $t$
\end{tabular}

\subsubsection{Discrete Quadratic Model}

\begin{tabular}{ll}
  $\mathbb{M}$ & Indices of all variables of the model \\
  $m$ & Mapping function of $\mathbb{I} \times \mathbb{T} \to \mathbb{M}$ \\
  $P_{i}$ & Vector of possible power output levels of unit $i$ \\
  $\delta_i$ & Difference between the non-zero power levels of $P_i$ \\
  $n_i$ & Defines number of elements in $P_i$: $| P_i | = 2^{n_i}$ \\
  $E$ & Energy Function of DQM \\
  $\gamma_c$ & Factor of ojective function \\
  $\gamma_s$ & Factor of startup and shutdown constraints \\
  $\gamma_d$ & Factor of demand constraint \\
  $p_{i, t}$ & Vector that signals the power output of unit $i$ at time $t$ \\
  $F_{i}$ & Vector of costs of corresponding power output level $P_{i}$ \\
  $S_i^{(0)}$ & Vector of startup or shutdown costs of unit $i$ at time $0$ \\
  $S_{i}$ & Matrix of startup and shutdown costs of unit $i$ \\
  $v_{\text{DQM}}$ & Number of variables of the model \\
  $q_{\text{DQM}}$ & Number of quadratic biases of the model
\end{tabular}

\subsubsection{Quadratic Unconstrained Binary Optimization}

\begin{tabular}{ll}
  $k \in \mathbb{N}_0, k < | P_i |$ & Indices of possible power levels of plant $i$ \\
  $\mathbb{M}'$ & Indices of all variables of the model \\
  $m'$ & Mapping function of $\I \times \T \times \{k \in \mathbb{N}_0 | k < | P_i |\} \to \mathbb{M}'$ \\
  $P_{i, k}$ & $k$th possible power output level of unit $i$ \\
  $E$ & Energy function of QUBO \\
  $\gamma_p$ & Factor of discretization constraints \\
  $p_{i, t, k}$ & Variable that signals the power output of unit $i$ at time $t$ at level $k$ \\
  $F_{i, k}$ & Costs of corresponding power output level $P_{i, k}$ \\
  $S_{i, k}^{(0)}$ & Startup or shutdown cost of unit $i$ at time $0$ at level $k$ \\
  $v_{\text{QUBO}}$ & Number of variables of the model \\
  $q_{\text{QUBO}}$ & Number of quadratic biases of the model
\end{tabular}


\chapter{Introduction}

The scheduling of conventional thermal power plants is crucial with the rise of renewable energies.
The Unit Commitment Problem (UCP) tackles this problem~\cite{Banos2011}.
Also, it can be used with wind energy and thermal power plants to minimize the cost of incorrect weather predictions.
These approaches are also called risk-based because they model the risk of the weather predictions being wrong~\cite{Chen2008, Abujarad2017}.

The number of deployed smart meters in the European Union was 99 million in 2018.
A study by the European Commission projected the number to reach 123 million by 2020~\cite{Vlachogiannis2019}.
With the increasing number of smart meters, the amount of information about energy consumption is growing.
The data is available with a time resolution that was not yet possible.
Due to this large amount of data, predicting future power consumption is possible on a house-to-house basis.
That allows for a much better overall prediction of the power consumption of towns or cities~\cite{Aiello2016, Basu2013}.
It presents the possibility to schedule thermal power plants much more precisely and with a higher time resolution.

In the future, communities with many private renewable energy sources might rely on distributed energy generation.
Distributed energy generation means that the energy produced by a consumer through their renewable energy sources gets added to the grid if the consumer is currently not using this energy~\cite{Aiello2016}.
Additionally, electric vehicles can be used as energy sources when plugged into a charger~\cite{Zhang2016}.
The addition of these energy sources gives additional degrees of freedom for energy procurement.

Figure \ref{figure:problem.sketch} is a sketch of the problem that the work considers.
It shows a town with offices and houses, as well as charging stations for electric vehicles.
There are $4$ power plants around the town that are all connected to it.

Optimizing the UCP takes a lot of time for large inputs.
Large inputs here means one of two things or both:
\begin{enumerate*}[label=(\roman*)]
  \item More units, i.e., power plants or
  \item A longer timeframe, over which to schedule the units.
\end{enumerate*}
UCPs are optimization problems and thus, most of the time formulated as Mixed-integer Non-linear Problems~\cite{Baldick1995}.
Precisely optimizing these problems is NP-complete~\cite{Li2005, Bienstock1996}.
The energy industry needs faster algorithms for large inputs to deal with the increasing amount of information.
These algorithms also have to be precise.

Quantum computing promises optimization algorithms with improved time complexity~\cite{Portnov2000, Ahuja1999, Gilliam2019, Ajagekar2020, Shaydulin2019}.
There are two types of quantum processors.
One of which is designed for optimizations and the other which is a more general quantum processor.

Today's quantum computers are described as Noisy Intermediate-Scale Quantum (NISQ) technology.
That is because the hardware is sensitive to noise and thus not able to run long computations.
Also, there are not many qubits in the latest quantum computers (intermediate-scale)~\cite{Leymann2020}.
That is why one can't run most quantum algorithms for practical problems.
Either there are not enough qubits to represent the variables, or the result is too noisy.
One alternative that still can give a speed-up for some problems are hybrid classical-quantum algorithms.
Such algorithms solve the problem at hand with classical computers and use quantum computers to speed up specific parts that a NISQ quantum computer can compute~\cite{Ajagekar2020, Shaydulin2019}.

This work explores the application of purely quantum and hybrid classical-quantum algorithms to the UCP.
It compares their performance in regards to computation time and accuracy of the result.
The goal is to find an algorithm that can accurately solve large UCPs faster than the current classical algorithms.

\begin{figure}[!ht]
  \centering
  \includegraphics[width=\textwidth]{01_Intro/problem_sketch.png}
  \caption{Sketch of Unit Commitment Problem}
  \label{figure:problem.sketch}
\end{figure}

\chapter{Fundamentals}

\section{Unit Commitment Problem} \label{intro:ucp}

The Unit Commitment Problem (UCP) generally is concerned with the scheduling or ''commitment''
of thermal power plants over a given time period.
Where thermal power plants can be coal, oil, gas or nuclear power plants.
The commitment of the power plants should be scheduled in such a way,
that the power demand or ''netwrok load'' at each moment of the time period
is satisfied by the power output of the power plants.
While satisfying the netwrok load, the fuel cost should be minimal.
The fuel cost depends on the commitment of each power plant at each time ($u_{i, t}$),
the power output of each power plant at each time ($p_{i, t}$),
the power plant specific cost function coefficients ($A_i$, $B_i$ and $C_i$),
as well as start-up and shut-down costs ($A_{i}^{U}$ and $A_{i}^{D}$).
Other than the netwrok load, the problem also consideres various other constraints of the power plants.
These include the minimum and maximum power output of each plant ($P_{min,i}$ and $P_{max,i}$),
the ramping constraints of each plant,
and the minimum up- and down-time.
\cite{Baldick1995}

In this paper, only the network load serves as a constraint to the commitment of the thermal power plants.
So it consists of the cost function $f_{i, t}$ that is to be minimized,
the network load constraint for each time $t$,
and the power output constraint for each unit $i$ and time $t$.

The UCP is non-linear in nature, because of the quadratic cost function $f_{i, t}$.
It also has floating-point variables as well as binary variables,
where binary variables can be represented as integer variables with domain $\{0, 1\}$.
The class of mixed-integer nonlinear problems (MINLP) is a class of optimization problems,
with a non-linear objective funtion,
integer and floating-point variables,
as well as various constraints on the variables.
This is the reason, that the UCPs are often formulated as MINLPs.
These models are non-convex in the case of the UCP, due to the binary nature of the commitment variables $u_{i, t}$.
\cite{Baldick1995, Abujarad2017}

\section{Mixed-Integer Nonlinear Problem}


\todo{Explain Conxept of MINLPs}

\section{Quantum Computing}

Quantum computers use quantum bits (qubits) instead of normal bits that classical computers use.
While a bit can be either in the state $0$ or $1$, a qubit can be in a so-called superposition of two states.
States of qubits are generally written using the Dirac-Notation, also known as the Bra-Ket-Notation
The two states, the superposition is a combination of, are called the basis.
The most basic basis is the basis $\{ \ket{0}, \ket{1} \}$.
The formula (\ref{formula:quantum.superposition}) describes such a superposition.
When one measures the state of the qubit, it collapses into one of the base states, which is the result of the measurement.
The probability of either basis state being measured is the square of the the complex amplitude of the basis state in the superposition.
\cite{Vedral1998}
\begin{subequations}
\begin{align}
  \label{formula:quantum.superposition}
  \ket{\phi} & = \alpha \cdot \ket{0} + \beta \cdot \ket{1}
  = \begin{pmatrix}
    \alpha \\ \beta
  \end{pmatrix}
  & \text{with } \alpha, \beta \in \mathbb{C}, \alpha^2 + \beta^2 = 1
  \\
  p_0 & = \alpha^2
\end{align}
\end{subequations}

Multiple qubits chained together form a quantum register.
In such a register, the measurement results of one qubit might depend on the measurement results of other qubits.
This is called entanglement.
The formula (\ref{formula:quantum.superposition.register}) describes a quantum register with $n$ qubits.
If one can't separate the formula into single qubits, the state is entangled.
The probability of measuring a specific state of the quantum register is given by the amplitude of that state squared.
\cite{Vedral1998}
\begin{subequations}
\begin{align}
  \label{formula:quantum.superposition.register}
  \ket{\Phi} & = \sum_{x \in \{0, 1\}^n} \alpha_x \cdot \ket{x}
  = \begin{pmatrix}
    \alpha_{0 \ldots 00} \\ \alpha_{0 \ldots 01} \\ \alpha_{0 \ldots 10} \\ \ldots \\ \alpha_{1 \ldots 11}
  \end{pmatrix}
  & \text{with } \forall x \in \{0, 1\}^n: \alpha_x \in \mathbb{C},
  \sum_{x \in \{0, 1\}^n} \alpha_x^2 = 1
  \\
  p_x & = \alpha_x^2
\end{align}
\end{subequations}

\input{02_Background/03_Quantum/01_annealing}
\input{02_Background/03_Quantum/02_gatebased}

\subsection{State of the art}

\subsubsection{Annealing-based Quantum Computers}

D-Wave is the leading manufacturer of Quantum Annealers.
Their newest model, ''Advantage'', has over $5, 000$ qubits and over $35, 000$ couplers.
The connectors connect the qubits in a $P_{16}$-Graph, also called ''Pegasus''.
This graph has the degree $16$, which means that every qubit has a connection with $16$ other qubits.
D-Wave released this model in 2020.
\cite{D-Wave2020, Zbinden2020}

The model before this, ''2000Q'', has over $2, 000$ qubits and over $6, 000$ couplers.
The connectors connect the qubits in a $C_{16}$-Graph, also called ''Chimera''.
This graph has the degree $6$, which means that every qubit has a connection with $6$ other qubits.
D-wave released that model in 2017.
\cite{D-Wave2020, Zbinden2020}

D-wave, therefore, increased their qubit number significantly in the last years.
The qubit count increased by $150\%$ in just $3$ years.
The coupler count increased by $580\%$ in the same time frame, although this number is misleading.
The more important metric is the qubit connectivity, the so-called ''degree'', which increased by $133\%$.

\begin{figure}[!h]
  \centering
  \includegraphics[width=0.7 \textwidth]{02_Background/dwave_qbits_history.png}
  \caption{Number of qubits in D-wave QPUs over the years \cite{D-Wave2018, D-Wave2020}}
  \label{figure:annealing.processors.history}
\end{figure}

Figure \ref{figure:annealing.processors.history} shows the increase of the number of qubits that are in D-Waves QPUs.
The number of qubits doubles every $2$ years.
These numbers remind of Moore's Law that dictated the advances in classical computing up until recently.
\cite{Theis2017}

\subsubsection{Gate-based Quantum Computers}

\todo{repeat with IBM as done for D-wave}

\section{Related Work}

Many methods for optimizing the UCP described in section \ref{fundamentals:ucp} exist in the literature.
\citeauthor{Abujarad2017} listed and compared those methods.
They listed the use of Mixed-integer Linear Programming as accurate but exponentially expensive.
All other methods give a sub-optimal solution~\cite{Abujarad2017}.

When dealing with cost functions of thermal power plants, they seldom are linear.
Most commonly, they are quadratic.
\citeauthor{Baldick1995} generalized the formulation of the UCP as Mixed-integer Non-linear Problems, where he assumed a quadratic cost function.

\citeauthor{Ajagekar2019} explore the utility of quantum computing for optimization problems in the energy domain.
They also considered the UCP.
They formulated a UCP as a QUBO and used D-Wave quantum hardware --- the D-Wave 2000Q, to be precise --- to optimize that model of the UCP.
Compared to the optimization on a classical computer using the Gurobi solver, the quantum sampler did perform a lot worse than the classical algorithm~\cite{Ajagekar2019}.

At the end of 2020, D-Wave released a hybrid solver that can solve Discrete Quadratic Models~\cite{DQMHybrid2020}.
The DQM model has a large advantage over the QUBO model when modeling problems with discrete or continuous variables.
The QUBO can only have binary variables, while the DQM can have discrete variables.
Modeling discrete variables takes many binary variables and many more quadratic biases to ensure only one binary variable corresponding to one discrete variable is $1$.
This phenomenon also appears in the QUBO formulation of the UCP by \citeauthor{Ajagekar2019}.

This work attempts to solve the UCP on annealing and gate-based quantum hardware using purely quantum and hybrid classical-quantum algorithms.
It considers the DQM formulation in addition to the QUBO formulation already tested by \citeauthor{Ajagekar2019}.
It also compares the performance of the proposed algorithms versus a classical algorithm.


\chapter{Approach}

\input{03_Approach/00_Classical/01_minlp}
\input{03_Approach/00_Classical/02_classical_implementation}

\section{Quantum Computing}

Quantum computers use quantum bits (qubits) instead of normal bits that classical computers use.
While a bit can be either in the state $0$ or $1$, a qubit can be in a so-called superposition of two states.
States of qubits are generally written using the Dirac-Notation, also known as the Bra-Ket-Notation
The two states, the superposition is a combination of, are called the basis.
The most basic basis is the basis $\{ \ket{0}, \ket{1} \}$.
The formula (\ref{formula:quantum.superposition}) describes such a superposition.
When one measures the state of the qubit, it collapses into one of the base states, which is the result of the measurement.
The probability of either basis state being measured is the square of the the complex amplitude of the basis state in the superposition.
\cite{Vedral1998}
\begin{subequations}
\begin{align}
  \label{formula:quantum.superposition}
  \ket{\phi} & = \alpha \cdot \ket{0} + \beta \cdot \ket{1}
  = \begin{pmatrix}
    \alpha \\ \beta
  \end{pmatrix}
  & \text{with } \alpha, \beta \in \mathbb{C}, \alpha^2 + \beta^2 = 1
  \\
  p_0 & = \alpha^2
\end{align}
\end{subequations}

Multiple qubits chained together form a quantum register.
In such a register, the measurement results of one qubit might depend on the measurement results of other qubits.
This is called entanglement.
The formula (\ref{formula:quantum.superposition.register}) describes a quantum register with $n$ qubits.
If one can't separate the formula into single qubits, the state is entangled.
The probability of measuring a specific state of the quantum register is given by the amplitude of that state squared.
\cite{Vedral1998}
\begin{subequations}
\begin{align}
  \label{formula:quantum.superposition.register}
  \ket{\Phi} & = \sum_{x \in \{0, 1\}^n} \alpha_x \cdot \ket{x}
  = \begin{pmatrix}
    \alpha_{0 \ldots 00} \\ \alpha_{0 \ldots 01} \\ \alpha_{0 \ldots 10} \\ \ldots \\ \alpha_{1 \ldots 11}
  \end{pmatrix}
  & \text{with } \forall x \in \{0, 1\}^n: \alpha_x \in \mathbb{C},
  \sum_{x \in \{0, 1\}^n} \alpha_x^2 = 1
  \\
  p_x & = \alpha_x^2
\end{align}
\end{subequations}

\input{02_Background/03_Quantum/01_annealing}
\input{02_Background/03_Quantum/02_gatebased}

\subsection{State of the art}

\subsubsection{Annealing-based Quantum Computers}

D-Wave is the leading manufacturer of Quantum Annealers.
Their newest model, ''Advantage'', has over $5, 000$ qubits and over $35, 000$ couplers.
The connectors connect the qubits in a $P_{16}$-Graph, also called ''Pegasus''.
This graph has the degree $16$, which means that every qubit has a connection with $16$ other qubits.
D-Wave released this model in 2020.
\cite{D-Wave2020, Zbinden2020}

The model before this, ''2000Q'', has over $2, 000$ qubits and over $6, 000$ couplers.
The connectors connect the qubits in a $C_{16}$-Graph, also called ''Chimera''.
This graph has the degree $6$, which means that every qubit has a connection with $6$ other qubits.
D-wave released that model in 2017.
\cite{D-Wave2020, Zbinden2020}

D-wave, therefore, increased their qubit number significantly in the last years.
The qubit count increased by $150\%$ in just $3$ years.
The coupler count increased by $580\%$ in the same time frame, although this number is misleading.
The more important metric is the qubit connectivity, the so-called ''degree'', which increased by $133\%$.

\begin{figure}[!h]
  \centering
  \includegraphics[width=0.7 \textwidth]{02_Background/dwave_qbits_history.png}
  \caption{Number of qubits in D-wave QPUs over the years \cite{D-Wave2018, D-Wave2020}}
  \label{figure:annealing.processors.history}
\end{figure}

Figure \ref{figure:annealing.processors.history} shows the increase of the number of qubits that are in D-Waves QPUs.
The number of qubits doubles every $2$ years.
These numbers remind of Moore's Law that dictated the advances in classical computing up until recently.
\cite{Theis2017}

\subsubsection{Gate-based Quantum Computers}

\todo{repeat with IBM as done for D-wave}


\chapter{Evaluation}
\label{evaluation}

\section{Data used for Validation}
\label{validation:data}

\subsection{Consumers}

\todo{List demand data, explain origin}

\subsection{Producers}

\todo{List data of plants}

\section{Performance of Classical Computers}

For the optimization of the Pyomo model, the Pyomo library runs a standard solver.
It formulates the problem in the mathematical modeling language AMPL
and calls a solver to optimize it.
The solver returns the solution, which Pyomo then reads~\cite{PyomoAMPL}.

This work uses the open-source solver Couenne for classical optimizations.
COIN-OR built this solver.
Couenne is short for ``Convex Over and Under ENvelopes for Nonlinear Estimation''.
It can find global minima of nonconvex MINLPs like the one that this work considers.
It is a branch and bound algorithm and utilizes linearization, bound reduction, and branching methods~\cite{Belotti2009, CoinorHome, CouenneRepo}.

This work performs the classical optimizations on the Excess Cluster of the HLRS, Stuttgart~\cite{ExcessHLRS, HLRS}.

The optimization of the problem is of high time complexity.
In the beginning, with small inputs, it takes little time.
As listed in table \ref{table:evaluation.classical.performance}, the optimization of $4$ power plants with $2$ units of time takes about $0.5$ seconds.
But as the input size grows, the time needed by the solver grows exponentially.
When the input size is doubled --- $4$ plants over $4$ units of time --- the optimization takes about $3.5$ seconds.
After adding another $2$ units of time, the optimization takes $15.5$ seconds.

As the number of inputs grows large, the optimization grows even faster than exponentially.
With a worst-case projection of the first $4$ data points, the time needed for optimization should multiply by $7$ with every additional $2$ units of time.
That means that the optimization of $4$ power plants over $14$ units of time should take about $0.5 \cdot 7^6 \approx 58,824$ seconds.
But the data shows that it took about $311,144.9$ seconds.

\begin{table}[ht]
  \centering
  \input{81_Tables/04_Validation/performance_classical_p4}
  \caption{Results of Classical Optimization with $4$ Power Plants}
  \label{table:evaluation.classical.performance}
\end{table}

The diagram \ref{figure:evaluation.classical.performance} highlights the fast growing computation time needed for the optimization.
Only the last two data points are distinguishable from the first data points because their computation time is very long compared to the first data points.
It shows the time needed for optimization dependent on the input size --- the number of time instances.

\begin{figure}[ht]
  \centering
  \includegraphics[width=0.75 \textwidth]{04_Validation/performance_classical_p4.png}
  \caption{Time Complexity of Classical Optimization with $4$ Power Plants}
  \label{figure:evaluation.classical.performance}
\end{figure}

\section{Performance of Annealing-based Quantum Computers}

\subsection{Implementation using DQM}

For the optimization using the Quantum Annealing hardware, the formulated DQM is sent to a hybrid solver at D-Wave.
The hybrid solver does not put the DQM on quantum hardware directly.
It classically searches the solution space.
The hybrid solver speeds up this process by using quantum hardware to determine promising regions to explore next.
\cite{DQMHybrid2020}

The optimization of the DQM using the hybrid solver for small problems seems to have linear time complexity.
Every time $10$ time instances are added to the problem size, the optimization takes about $0.1$ seconds longer.
The results of the optimization and the time needed for the optimization are listed in table \ref{table:validation.annealing.performance}.
The optimization of these problems takes no longer than $5.6$ seconds each.

\begin{table}[ht]
  \centering
  \input{81_Tables/04_Validation/performance_annealing_p4.tex}
  \caption{Results of Annealing Optimization with $4$ Power Plants}
  \label{table:validation.annealing.performance}
\end{table}

In figure \ref{figure:validation.annealing.performance} the time complexity is displayed clearer.
The time needed for optimizing problems with less than $15$ time instances also seems to be constant.
After that, the linear time complexity is clearly visible.

When the problem size increases, the time complexity seems to be worse than linear.
This becomes clear in figure \ref{figure:validation.annealin.performance.extended}.
The time taken by the hybrid solver to optimize the problem grows linearly, but at a higher rate starting from $80$ time instances.
After this increase of slope, the slope stays the same until the problem size reaches $200$ in the time dimension.
I did not test the optimization of problems with more than $200$ time instances on the D-Wave hardware.

\begin{figure}
  \begin{subfigure}[b]{0.5 \textwidth}
    \centering
    \includegraphics[width=\textwidth]{04_Validation/performance_annealing_p4.png}
    \caption{With up to $50$ Loads}
    \label{figure:validation.annealing.performance}
  \end{subfigure}
  \begin{subfigure}[b]{0.5 \textwidth}
    \centering
    \includegraphics[width=\textwidth]{04_Validation/performance_annealing_p4_extended.png}
    \caption{With up to $200$ Loads}
    \label{figure:validation.annealin.performance.extended}
  \end{subfigure}
  \caption{Time Complexity of Annealing Optimization with $4$ Power Plants}
\end{figure}

This approach does not produce optimal solutions for the UCP.
The summed power output of the single power plants does not meet the power demand because the DQM underestimates the demand part of the UCP.
After retrieving the optimal DQM solution and translating it to a UCP solution, the program adjusts the power outputs of the power plants.
It does so as described in section \ref{approach:quantum.read.solution}.
This leads to a non-optimal solution of the UCP.

\subsubsection{Implementation using QUBO}

\todo{List performance results of QUBO on hybrid dwave solver}

\section{Performance of Gate-based Quantum Computers}

The implementation of the QUBO as described in section \ref{approach:gate.implement} of the UCP builds a quantum circuit for the first iteration of the optimization algorithm described in section \ref{backg:quantum.grover.optimization}.
Unfortunately, this quantum circuit is too big to be executed on any of IBM's public quantum computers for the smallest UCP that this work considers.

The number of qubits needed just for the QUBO variables is given by the formula (\ref{formula:qubo.num.qubits}).
This is considered the first register of the quantum computer.
In the case of the smallest UCP this work considers with the $4$ power plants listed in section \ref{table:validation.data.plants} and $2$ time instances, the number of qubits would be $\left( 2^6 + 2^6 + 2^7 + 2^8 \right) \cdot 2 = 1024$.
With every additional time instance of the UCP, another $1024$ qubits are needed.

Additionally to these qubits, the quantum circuit needs a second register for storing the result of the QUBO dependent on the QUBO variables of the first register.
This number also grows with the QUBO size.

This means that the gate-based quantum computers right now do not have enough qubits to handle the problems this work considers.

\section{Comparison of Performances}
\label{validation:comparison}

\todo{Compare performances of classical, annealing-based and gate-based Computers}


\chapter{Conclusion}

\todo{Formulate a conclusion}


\printbibliography


\pagestyle{empty}
\renewcommand*{\chapterpagestyle}{empty}
\Versicherung
\end{document}

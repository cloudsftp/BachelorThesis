This work proposes an approach to reformulate UCPs with inter-time dependencies as problems that quantum computers can solve.
DQMs and QUBOs are the names of the optimization problems that this approach produces.
Gate-based quantum computers can't optimize the resulting QUBOs because they lack qubits.
Annealing-based quantum computers can't optimize the resulting QUBOs directly on the QPU because they lack connections between their qubits.
But hybrid approaches using both classical computing and quantum computing can optimize the resulting QUBOs and DQMs.

This work compares the performance of hybrid annealing-based optimization of DQMs with the classical optimization of MINLPs for real-world like UCPs.
It is shown that the hybrid algorithm has a big computational advantage regarding the time it needs to optimize the UCP.
But also, it has a computational disadvantage regarding the quality of the result.
That leads to a trade-off between computation time and optimality of the result.

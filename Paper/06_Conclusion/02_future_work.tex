\subsection*{Future Work}


The conversion of the MINLP formulation of the UCP to a DQM formulation is not optimal.
While converting the UCP to a DQM, the proposed approach loses the ability to find the actual optimal solution to the UCP.
This work shows this experimentally in section \ref{evaluation:comparison}.

There might be a better way to formulate a DQM for the UCP.
That way must conserve the ability to find the actual optimal solution to the UCP.
That way could achieve the same computational advantage as demonstrated in section \ref{evaluation:comparison} if the size of the DQMs is about the same.
The size of the DQM depends on the number of variables, their possible values, and the number of biases.

Another approach to mitigate the non-optimality would be taking the result and computing the details using a classical algorithm.
The classical algorithm for optimizing MINLPs this work considers, ``Couenne'', can have a starting point for the optimization.
The idea is to use the hybrid DQM sampler's result as the starting point.

Yet another promising approach is to find a hybrid algorithm for optimizing MINLP problems directly.
Then the reformulation as a DQM would not be necessary.
\citeauthor{Ajagekar2020} demonstrated hybrid algorithms for similar mathematical optimization problems like the Mixed-integer Linear Problem (MILP).

Such a general-purpose hybrid MINLP solver would help not only the energy industry but also many other industries.
There exists an MINLP formulation for almost every real-world industry problem~\cite{Belotti2009}.
Thus most real-world industry problems could be solved using such a hybrid MINLP solver.

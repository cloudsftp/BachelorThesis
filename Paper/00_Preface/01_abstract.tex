\section*{Abstract}

This paper explores the use of Quantum Computing for the scheduling of thermal power plants.
With the rise of Smart Meters, the energy industry can forecast the energy demand more precisely and with better time resolution.
This increased time resolution increases the size of the mathematical models used to schedule the power plants.
Also, the rise of renewable energy sources further increases the size of the models.
Current techniques for the precise optimization of these models are computationally expensive and have high time complexity.
They involve Mixed-integer Non-linear Problems.
Quantum Computing is a powerful tool for optimizations.
The proposed method reformulates the Mixed-integer Non-linear Problem into a Discrete Quadratic Model.
A hybrid classical-quantum algorithm then optimizes this model.
The method decreases the time complexity but also loses some precision in finding the true optimum.
This paper compares its performance and precision with a classical open-source algorithm used for optimizing Mixed-integer Non-linear Problems.
The comparison involves running both algorithms on real-world data.

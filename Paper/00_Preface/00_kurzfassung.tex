\section*{Kurzfassung}

Diese Arbeit erkundet die Nützlichkeit des Quanten Computings für das Planen der Stromgenerierung thermischer Kraftwerke.
Da immer mehr Smart Meter, also smarte Stromzähler, in Haushalten verbaut werden, kann die Energieindustrie den Energieverbrauch genauer und mit einer höheren Zeitauflösung vorhersagen.
Dadurch werden die mathematischen Modelle größer.
Der Anstieg von Erneuerbaren Energien trägt auch zur weiteren Vergrößerung der Modelle bei.
Bisherige Optimierungen dieser Planungen involviert sogenannte Mixed-integer Non-linear Problems.
Für die Optimierung dieser Modelle sind nur Algorithmen mit einer sehr hohen Zeitkomplexität bekannt.
Die hier vorgestellte Methode formuliert das Problem zu einem Diskreten Quadratischen Modell.
Diese wird dann von einem hybriden klassisch-quanten Algorithmus optimiert.
Diese Methode reduziert die Zeitkomplexität aber reduziert gleichzeitig die Genauigkeit des Ergebnisses.
Diese Arbeit vergleicht die Methode mit einem klassischen open-source Algorithmus der zur Optimierung von Mixed-integer Non-linear Problems genutzt wird.
Beide Algorithmen werden auf realitätsnahen Daten getestet.

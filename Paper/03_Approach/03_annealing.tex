\section{Implementation for Annealing-based Quantum Computers}

The MINLP formulation of the UCP is converted to a Discrete Quadratic Model (DQM).
As mentioned in section (\ref{backg:annealing}), this model is the extension
of Binary Quadratic Models (BQM) or Quadratic Unconstrained Binary Optimization model (QUBO).

\subsubsection{Mapping Variables}

The model, as seen in section (\ref{backg:annealing}), takes in a vector rather than a 2-dimensional matrix.
That's why the indices of the variables $p_{i,t}$ have to be mapped, such that the result are just as many variables $p_{l}$.

\begin{align}
  k:
  \quad
  \mathbb{I} \times \mathbb{T} \to \mathbb{K}:
  \quad
  k(i, t) = i \cdot | \mathbb{T} | + t
\end{align}

In the model below, $p_{i,t}$ is used, but this is to show, that

\begin{subequations}
\begin{align}
  &
  \forall i \in \mathbb{I}, \forall t, t' \in \mathbb{T}, t < t'
  &
  \Rightarrow k(i, t) < k(i, t')
  \\
  \land \quad
  &
  \forall t \in \mathbb{T}, \forall i, j \in \mathbb{I}, i < j
  &
  \Rightarrow k(i, t) < k(j, t)
\end{align}
\end{subequations}

When implementing the DQM, the discrete variables $p_{i, t}$ are added in the order specified by $k(i, t)$.
This makes sure that the quadratic biases are added in the right orientation.

\subsubsection{Discretizising Variables}

The power levels have to be discretized, since the model, respectively the variables, are discrete.
To maximize the utilization of qbits on the QPU, every variable $p_{i, t}$ can have one of $2^{n_i}$ values.
$n_i$ is chosen so that the difference $\delta_i$ between the non-zero power levels represented by the values of $p_{i, t}$
is $5 < \delta_i \leq 10$.
The power levels of the discrete values are represented as $P_i$.
Their values are:

\begin{align}
  P_i =
  \quad \begin{pmatrix}
    0 \\
    P_{min, i} \\
    P_{min, i} + \delta_i \\
    P_{min, i} + 2 \cdot \delta_i \\
    \vdots \\
    P_{max, i}
  \end{pmatrix}
\end{align}

This automatically enforces the constraint (\ref{formula:minlp.power}) that ensures the power limits to stay inside the limits of every power plant.
This constraint thus is not found in the energy function (\ref{formula:dqm.start.obj}-\ref{formula:dqm.start.demand}).

\subsection{Formulating the DQM}

The formula we start with (\ref{formula:dqm.start.obj}-\ref{formula:dqm.start.demand})
includes the objective function and every constraint of the MINLP formulation (\ref{formula:minlp}).
Except for the constraint (\ref{formula:minlp.power}) that makes sure the output of every power plant is within its limits.
This is already enforced by the discretization of the power levels.

The formula is not in the standard form (\ref{formula:dqm.form}).
To achieve this form, the last summand has to be transformed and the linear parts and quadratic parts have to be combined into groups.

\begin{subequations}
\begin{align}
  E(p) =
  &  \quad \gamma_c \sum_{i, t} F_i p_{i, t}
  \label{formula:dqm.start.obj} \\
  + & \quad \gamma_s \sum_i \left(
      u_{i, -1} S_i^U p_{i, 0}
      + \sum_{t > 0} S_i \left( p_{i, t-1} \otimes p_{i, t} \right)
    \right)
  \label{formula:dqm.start.startup} \\
  + & \quad \gamma_d \sum_t \left( \left( \sum_i P_i p_{i, t} \right) - l_t \right)^2
  \label{formula:dqm.start.demand}
\end{align}
\end{subequations}

\subsubsection{Objective Function}

The first summand of the energy function represents the objective function (\ref{formula:minlp.obj}) without the startup and shutdown costs.
The objective function simply takes the sum of the costs of operating the power plants at the level they are operating at at every time instance.
This can be modeled by a vector that holds the cost of operating a power plant at the power lavel of the corresponding value.
It is achieved by applying the cost function used in the objective function (\ref{formula:minlp.obj}) element wise to $P_i$:

\begin{align}
  F_i = \quad \begin{pmatrix}
    0 \\
    A_i + B_i \cdot P_{min_i} + C_i \cdot \left( P_{min, i} \right)^2 \\
    A_i + B_i \cdot \left( P_{min_i} + \delta_i \right) + C_i \cdot \left( P_{min, i} + \delta_i \right)^2 \\
    A_i + B_i \cdot \left( P_{min_i} + 2 \cdot \delta_i \right) + C_i \cdot \left( P_{min, i} + 2 \cdot \delta_i \right)^2 \\
    \vdots \\
    A_i + B_i \cdot \left( P_{max_i} \right) + C_i \cdot \left( P_{max, i} \right)^2 \\
  \end{pmatrix}
\end{align}

\subsubsection{Startup and Shutdown Costs}

\todo{Resolve startup and shutdown costs part}

\subsubsection{Demand}

\todo{Resolve demand part}

\subsubsection{Result}

\begin{subequations}
\begin{align}
  E(p) = \quad
  &
  \gamma_c \sum_{i, t} F_i p_{i, t}
  + \gamma_s \sum_i \left(
      u_{i, -1} S_i^U p_{i, 0}
      + \sum_{t > 0} S_i \left( p_{i, t-1} \otimes p_{i, t} \right)
    \right)
  \\ &
  + \gamma_d \sum_t \left(
    L_t^2
    + \sum_i \left( P_i p_{i, t} \right)^2 - l_t \sum_i P_i p_{i, t}
    + \sum_{i < j} \left( P_i p_{i, t} \right) \left( P_i p_{j, t} \right)
  \right) \\
  = \quad
  &
  \gamma_s \sum_i \sum_{t > 0} S_i \left( p_{i, t-1} \otimes p_{i, t} \right)
  + \gamma_d \sum_t \sum_{i < j} \left( P_i \otimes P_j \right) \left( p_{i, t} \otimes p_{j, t} \right)
  \\ &
  + \sum_{i, t} \left(
    \gamma_c F_i + \gamma_d \left(P_i^2 - l_t P_i \right)
  \right) p_{i, t}
  + \sum_i u_{i, -1} S_i^U p_{i, 0}
  \\ &
  + \sum_t l_t^2
\end{align}
\end{subequations}

\todo{Explain where scalar products are used}

\subsection{Implementation}

\todo{Inplementation of a DQM}

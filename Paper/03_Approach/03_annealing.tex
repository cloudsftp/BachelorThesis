\section{Implementation for Annealing-based Quantum Computers}

The MINLP formulation of the UCP is converted to a Discrete Quadratic Model (DQM).
As mentioned in section (\ref{backg:annealing}), this model is the extension
of Binary Quadratic Models (BQM) or Quadratic Unconstrained Binary Optimization model (QUBO).

\subsubsection{Mapping Variables}

The model, as seen in section (\ref{backg:annealing}), takes in a vector rather than a 2-dimensional matrix.
That's why the indices of the variables $p_{i,t}$ have to be mapped, such that the result are just as many variables $p_{j}$.

\todo{Mapping indices}

\subsubsection{Discretizising Variables}

Since the model is discrete, respectively the variables, the power levels have to be discretized.

\todo{Discretizise power levels}

Automaticcaly enforces constraint (\ref{formula:minlp.power}).

\subsection{Formulating the DQM}

The formula we start with (\ref{formula:dqm.start}) includes the objective function and every constraint of the MINLP formulation (\ref{formula:minlp}).
Except for the constraint (\ref{formula:minlp.power}) that makes sure the output of every power plant is within its limits.
The formula is not in the standard form (\ref{formula:dqm.form}).
To achieve this form, every summand of the formula has to be resolved.

\begin{align} \label{formula:dqm.start}
  E(p) = \gamma_c \sum_{i, t} f_i p_{l(i, t)}
  + \gamma_s \sum_i \sum_{t > 0} S_i p_{l(i, t-1)} p_{l(i, t)}
  + \gamma_d \sum_t \left( \left( \sum_i P_i p_{l(i, t)} \right) - L_t \right) ^2
\end{align}

\subsubsection{Objective Function}

\todo{Resolve objective function part}

\subsubsection{Startup and Shutdown Costs}

\todo{Resolve startup and shutdown costs part}

\subsubsection{Demand}

\todo{Resolve demand part}

\subsubsection{Result}

\begin{align}
  &
  E(p) = \gamma_c \sum_{i, t} f_i p_{l(i, t)}
  + \gamma_s \sum_i \sum_{t > 0} S_i p_{l(i, t-1)} p_{l(i, t)}
  \\
  &
  + \gamma_d \sum_t \left(
    \sum_{i < j} P_i P_j p_{l(i, t)} p_{l(j, t)}
    + \sum_i P_i^2 p_{l(i, t)} - L_t \sum_i P_i p_{l(i, t)}
    + L_t^2
  \right)
\end{align}

\todo{simplify}

\subsection{Implementation}

\todo{Inplementation of a DQM}

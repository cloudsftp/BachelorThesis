\section{Unit Commitment Problem as Quadratic Unconstrained Binary Model}

The formulation of the Quadratic Unconstrained Binary Model (QUBO) is based on the DQM formulated in section \ref{approach:annealing.formulate}.
Instead of a single discrete variable for every power plant $i$ and time instance $t$, the QUBO has an array of binary variables for every possible power level in $P_i$ for every power plant $i$ and time instance $t$.

\subsubsection{Mapping Variables}

As before in section \ref{approach:annealing.formulate}, the variable indices are mapped to a single index to make sure the form of the QUBO is correct.
The mapping is extended by a third index $k$ corresponding to the power level.

\begin{align}
  m':
  \quad
  \mathbb{I}
  \times \mathbb{T}
  \times \{k \in \mathbb{N} | k \leq |P_i|\} \to \mathbb{K}':
  \quad
  m'(i, t, k) = \left( |\mathbb{T}| \sum_{j=0}^{i-1} |P_j| \right)
  + t \cdot |P_i|
  + k
\end{align}

In the model below, $p_{i,t,k}$ is used, but this is to show that

\begin{subequations}
\begin{align}
  &
  \forall i \in \mathbb{I}, t \in \mathbb{T}, k, l \in \mathbb{N}, k < l < |P_i|:
  &
  m'(i, t, k) < m'(i, t, l)
  \\
  \land \quad
  &
  \forall i \in \mathbb{I}, t, t' \in \mathbb{T}, k \in \mathbb{N}, k < |P_i|, t < t':
  &
  m'(i, t, k) < m'(i, t', k)
  \\
  \land \quad
  &
  \forall i, j \in \mathbb{I}, t  \in \mathbb{T}, k, l \in \mathbb{N}, k < |P_i|, l < |P_j|, i < j:
  &
  m'(i, t, k) < m'(j, t, l)
\end{align}
\end{subequations}

When implementing the BQM, the discrete variables $p_{i, t, k}$ are added in the order specified by $m'(i, t, k)$.
This order ensures that the applying of quadratic biases does not violate the form of the BQM.
When adding a quadratic bias, the first argument is always a variable of lower order.

\subsubsection{Discretizising Power Levels}

This approach discretizes the power levels just like in section \ref{approach:annealing.discretize}.
But this time the values are not a array but single values.
They are defined as

\begin{align}
  P_{i, k} = \begin{cases}
    0 & , k = 0 \\
    P_{min, i} + (k - 1) \cdot \delta_i & , else
  \end{cases}
\end{align}

$P_{i, k}$ is defined for all $i \in \mathbb{I}, k \in \mathbb{N}, k < 2^{n_i}$.
Where $n_i$ is chosen as in section \ref{approach:annealing.discretize}.
Thus $P_{i, 2^{n_i} - 1} = P_{max, i}$.

This automatically enforces the constraint (\ref{formula:minlp.power}) that ensures the power limits to stay inside the limits of every power plant.
This constraint thus is not found in the energy function (\todo{link parts of function}).

\section{Implementation for Annealing-based Quantum Computers}

The implementation of the DQM uses D-Waves \texttt{dwave-ocean-sdk}.
\cite{OceanSDKDoc}
To be more specific, it uses the class \texttt{DiscreteQuadraticModel} that is part of the \texttt{dimod} package.
\cite{DQMDoc}
The source code of the SDK is available online.
\cite{OceanSDKGit}

\subsection*{Variables}

The first step is to instantiate a DQM with discrete variables.
One discrete variable is added to the model for every $p_{i, t}$ of the DQM defined in section \ref{approach:annealing.formulate}.
The size of the discrete variable $p_{i, t}$ --- the number of possible values --- is $| P_i |$.

\subsection*{Quadratic Biases}

There are two types of quadratic biases.
One category is the startup and shutdown biases, and the other is the biases resulting from the demand constraints.

The startup and shutdown biases are different for every plant but do not change over time.
The program computes them once for every plant.
Then it applies them to variables that correspond to the same plant but correspond to adjacent time instances.
So $S_i$ is computed and then applied to variables $p_{i, t-1}$ and $p_{i, t}$ for all $t > 0$.

The demand biases are different for every pair of plants targeted by the bias but do not change over time.
The program computes them once for every pair of plants.
Then it applies them to the variables that correspond to the same time instance but correspond to each one of the targeted plants.
So $P_i \otimes P_j$ is computed and then applied to the variables $p_{i, t}$ and $p_{j, t}$ for all $i < j$.

The two categories of quadratic biases do not overlap.
This is because the startup and shutdown biases are applied along the time axes,
and the demand biases are applied along the plant axes, which is orthogonal to the time axes.

\subsection{Adjustment of Solution}
\label{approach:annealing.implementation.adjust}

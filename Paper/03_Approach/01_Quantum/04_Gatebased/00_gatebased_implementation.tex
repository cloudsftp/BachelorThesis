\section{Implementation for Gate-based Quantum Computers}

\todo{explain implementation using Qiskit}

\subsection{Adjusting Solution}

Because the QUBO is designed based on the DQM it will also underestimate the power demand.
The program will use the same algorithm (\ref{approach:annealing.implementation.adjust.algorithm}) as used for the DQM before to correct the error.
The algorithm is described in section \ref{approach:annealing.implementation.adjust}.

Before adjusting the result, the algorithm first has to read out the results from the sample.
It can happen that for a single power plant at a single time instance, multiple binary variables for different power levels are $1$.
The algorithm then has to choose one of the possible power levels where the binary variable is $1$.
For this the algorithm \ref{approach:gatebased.implementation.adjust.algorithm} is used.

\begin{lstlisting}[
  caption={Choosing Power Levels from the QUBO},
  label={approach:gatebased.implementation.adjust.algorithm},
  language=Python
]
value_indices: List[int] = []
for k in range(len(self.P[i])):
  if result[self.p[i][t][k].name] == 1:
    value_indices.append(k)

value: float = 0
num_indices: int = len(value_indices)
if num_indices > 0:
  value = self.P[i][value_indices[(int) (num_indices / 2)]]
  if num_indices > 1:
    debug_msg('Warning: {} possible power levels for plant {} detected'.format(num_indices, i))
\end{lstlisting}

The program always chooses the median value.

\subsection{Implementation of QUBO for Annealing Hardware}
\label{approach:annealing.implement.qubo}

The implementation of the QUBO for the annealing hardware uses the UQO-client.
\cite{UQOPyPI, UQOGitHub}
First, it writes the biases into a python \texttt{dict}.
Then the class \texttt{Qubo} of the UQO-client builds the \texttt{dwave-ocean-sdk} \texttt{BinaryQuadraticModel}-instance that the UQO-client can then send to the UQO-server.

\subsubsection{Variables}
\label{approach:annealing.implement.qubo.variables}

The variables in this implementation are implicit.
The indices for the python \texttt{dict} are pairs of integers where the integers are the IDs of the variables.
Each ID is computed with the mapping function $m'$ of formula (\ref{formula:qubo.mapping}).

\subsubsection{Quadratic Biases}
\label{approach:annealing.implement.qubo.quadratic}

In this case, there are $3$ types of quadratic biases
--- biases resulting from either
\begin{enumerate*}[label=(\roman*)]
  \item Startup and Shutdown costs, or
  \item Demand constraints, or
  \item Discretization constraints
\end{enumerate*}

The startup and shutdown costs are different for every power plant but do not change over time.
The program iterates over all power plants $i$ and time instances $t > 0$ and applies the startup and shutdown costs.
The program iterates over every power output value index $0 \leq k < 2^{n_i}$ for the power plant $i$.
It adds the quadratic bias $A_U$ for the pair $(p_{i, t-1, 0}, p_{i, t, k})$ and the quadratic bias for the pair $(p_{i, t-1, k}, p_{i, t, 0})$

The demand constraints are different for every pair of plants but do not change over time, like the startup and shutdown costs.
The program iterates over all pairs of power plants $(i, j)$ where $i < j$ and power output value indices $0 \leq k < 2^{n_i}$ and $0 \leq l < 2^{n_j}$.
Then it iterates over all time instances $t$ and applies the quadratic constraint $\gamma_d P_{i, k} * P_{j, l}$ to the pair $(p_{i, t, k}, p_{j, t, l})$
Notice that the quadratic constraint is only computed once for every pair of power plants $(i, j)$ and pair of power output values $(k, l)$.

The discretization constraint is the same for every plant and time instance.
The program iterates over all power plants $i$ and time instances $t$.
Then it iterates over all pairs of power output value indices $(k, l)$ where $0 \leq k < 2^{n_i}, 0 \leq l < 2^{n_j}$ and $k < l$ and apllies the quadratic constraint $\gamma_d$ to the pair $(p_{i, t, k}, p_{i, t, l})$.

\todo{this paragraph is not correct. no quadratic biases are in conflict}
In this case, the discretization biases and the demand constraining biases overlap.
The program adds them together if it applies two different quadratic constraints to the same pair.

\subsubsection{Linear Biases}
\label{approach:annealing.implement.qubo.linear}

The linear biases depend on the fuel cost and the demand constraints.
That's why they change for every plant $i$, every time instance $t$, and every possible power level $k$.
Just as with the implementation of the DQM in section \ref{approach:annealing.implement.dqm.linear}, the linear biases of variables of the form $p_{i, 0, k}$
--- meaning it corresponds to the first time instance ---
depend on the startup and shutdown costs and the initial state of the power plant $i$.
The formula (\ref{formula:qubo.result.linear}) captures the computation of the linear biases.

The program iterates over all power plants $i$, time instances $t$, and possible power levels $0 \leq k < 2^{n_i}$.
It calculates the linear bias as specified in formula (\ref{formula:qubo.result.linear}) for all indices.
Then it applies the linear bias to the variable $p_{i, t, k}$.

As mentioned in section \ref{approach:annealing.implement.qubo}, the program first builds the QUBO as a python \texttt{dict}.
The keys are pairs of integers $(i, j)$ with $i \neq j$ for quadratic biases.
For linear biases, the keys are pairs with identical integers $(i, i)$.

\subsubsection{Constant Bias}
\label{approach:annealing.implement.qubo.constant}

The implementation ignores the constant bias.
In this case it is defined by the formula (\ref{formula:qubo.result.constant}).
The constant bias does not change the input values that produce a minimum.
It only changes the value of that minimum.

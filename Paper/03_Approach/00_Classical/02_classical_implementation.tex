\section{Implementation on Classical Computers}

For the implementation of UCPs for classical computers
the MINLP formulation is implemented using the Python library Pyomo.
Pyomo is open-source and supports the formulation of various optimization problems,
particularly mixed-integer nonlinear problems.
It also supports performing the optimization of the problems
with standard open-source or commercial solvers.
\cite{hart2011pyomo}

The program is given a UCP object, that holds information
about the power demands and the available power plants.
It then creates a ''ConcreteModel'' from this information.
This model of the MINLP formulation of the UCP
has three $| \mathbb{I} | \times | \mathbb{T} |$ matrices as variables.
The first two are the variables fo the commitment $u_{i, t}$
and the power output $p_{i, t}$ for each power plants at each time instance.
These are the free variables of the model that will be adjusted so that the objective function is minimized.
The last matrix is for $s_{i, t}$, the startup and shutdown costs.
These are not free but depend on the data from the UCP object.

\subsubsection{The Objective Function}

The formula (\ref{formula:minlp.obj}) is modeled using a Objective object provided by the Pyomo framework.
It depends on the model variables for $u_{i, t}$, $p_{i, t}$ and $s_{i, t}$.
Where $u_{i, t}$ and $p_{i, t}$ are free variables and $s_{i, t}$ are defined through the model.
The coefficients $A_i$, $B_i$ and $C_i$ are taken from the UCP object.

The startup and shutdown costs are defined through a method called ''disjunctive programming''.
This method makes it possible to formulate ''either-or'' conditions
and tie the value of one variable to the value of other variables.
\cite{Balas1983}

With this method the value of $s_{i, j}$ set to either $A^U_i$, $A^D_i$ or $0$
depending on the values of $u_{i, t}$ and $u_{i, t-1}$.
This relation is defined mathematically by the constraint (\ref{formula:minlp.updowncost}).
And it is modeled by giving the model three disjunct events for each plant $i$ and time instance $t$:
\begin{subequations}
\begin{align}
  u_{i, t} < u_{i, t-1} \quad & \land & \quad S_{i, t} = A^U_i \\
  u_{i, t} > u_{i, t-1} \quad & \land & \quad S_{i, t} = A^D_i \\
  u_{i, t} = u_{i, t-1} \quad & \land & \quad S_{i, t} = 0
\end{align}
\end{subequations}
If $t = 0$, $u_{i, t-1}$ is defined as the initial commitment state of the power plant $i$.
The first part of the events is the condition and the second part defines the value of $s_{i, t}$.
Since the conditions are mutually exclusive,
only one of the events will become reality.
This means that $s_{i, t}$ is set to the value that is conjugated with the true condition.

Now there are two constraints left to model.

The first one is the constraint (\ref{formula:minlp.load}).
It makes sure that the power demand is met by all the power plants at every time instance $t$.
In the model a constraint list of length $|\mathbb{T}|$ is generated.
Each element of the list makes sure that the sum of all power outputs at its corresponding time instance
is equal to the power demand at that time instance.

The last constraint is the constraint (\ref{formula:minlp.power}).
It makes sure that the power output of every power plant is within its limits at every time instance.
In the model a $|\mathbb{I}| \times |\mathbb{T}|$ constraint matrix is generated.
Each element of the matrix makes sure the power output of the corresponding power plant
at the corresponding time instance is within the limits of the power plant.

For a model with disjunctive programming to be solvable by a standard solver,
it has to be transformed.
In this paper, the Big-M transformation is used.
\todo{Explain Big-M}
\cite{Trespalacios2015}

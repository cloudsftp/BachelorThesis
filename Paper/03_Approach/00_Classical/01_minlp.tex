\section{Unit Commitment Problem as Mixed-Integer Nonlinear Problem}

The UCP is formulated as a MINLP for classical computers to solve,
as mentioned in section \ref{backg:ucp}.
The equations (\ref{formula:minlp}) represent the formulated MINLP.

\begin{subequations}
\begin{align}
  \min_{u, p} \quad &
  \sum_{i \in \mathbb{I}, t \in \mathbb{T}} f_{i, t}
  = \sum_{i \in \mathbb{I}, t \in \mathbb{T}}
    u_{i, t} (A_i + B_i p_{i, t}' + C_i (p_{i, t}')^2) + s_{i, t}
  \label{formula:minlp.obj} \\
  \text{s.t.} \quad & l_t = \sum_{i \in \mathbb{I}} u_{i, t} p_{i, t}' \quad &
  \forall t \in \mathbb{T}
  \label{formula:minlp.load} \\
  &
  P_{min, i} \leq p_{i, t}' \leq P_{max, i} \quad &
  \forall i \in \mathbb{I}, t \in \mathbb{T}
  \label{formula:minlp.power} \\
  &
  s_{i, t} = \begin{cases}
    A_i^U & \text{if } u_{i, t} > u_{i, t-1} \\
    A_i^D & \text{if } u_{i, t} < u_{u, t-1} \\
    0 & \text{else}
  \end{cases} \quad &
  \forall i \in \mathbb{I}, t \in \mathbb{T}
  \label{formula:minlp.updowncost}
\end{align}
\label{formula:minlp}
\end{subequations}

\subsubsection{Objective Function}

The formula (\ref{formula:minlp.obj}) represents the global cost function.
This function has to be minimized and is called the ''objective function''.
It is the sum of all cost functions $f_{i, t}$ of the individual power plants at every time step.

The individual cost funtions $f_{i, t}$ take $u_{i, t}$ and $p_{i, t}'$ as free variables.
$A_i, B_i$ and $C_i$ are constant for every power plant.
$f_{i, t}$ also depend on $s_{i, t}$ or equation (\ref{formula:minlp.updowncost}),
which evaluates to $A_i^U, A_i^D$ or $0$ depending on the free commitment variables $u_{i, t}$ and $u_{i, t-1}$.
$A_i^U$ and $A_i^D$ are also constant for every power plant,
as well as $u_{i, -1}$.

$A_i, A_i^U$ and $A_i^D$ are the linear coefficients of the cost function,
because they are not a factor of $p_{i, t}$.
$A_i$ is the constant cost when the power plant is active
and is added to $f_{i, t}$ if $u_{i, t} = 1$.
$A_i^U$ is the cost of starting the power plant
and is added to $f_{i, t}$ if $u_{i, t} > u_{i, t-1}$.
$A_i^D$ is the cost of shutting down the power plant
and is added to $f_{i, t}$ if $u_{i, t} < u_{i, t-1}$.
If $t = 0$ this would lead to problems, because $u_{i, t-1} = u_{i, -1}$ would be undefined.
That's why the initial state of the power plant $u_{i, -1}$ is also constant.
$B_i$ is the linear coefficient of the cost function,
because it is a factor of $p_{i, t}'$.
It is multiplied by $p_{i, t}'$ and added to $f_{i, t}$ if $u_{i,  t} = 1$.
$C_i$ is the quadratic coefficient of the cost function,
because it is a factor of $p_{i, t}^2$.
It is multiplied by $p_{i, t}^2$ and added to $f_{i, t}$ if $u_{i,  t} = 1$.

So the obbjective function takes the free variables
$
(u_{i, t})_{i \in \mathbb{I}, t \in \mathbb{T}},
(p_{i, t}')_{i \in \mathbb{I}, t \in \mathbb{T}}
$ and the constants $
(A_i)_{i \in \mathbb{I}},
(A_i^U)_{i \in \mathbb{I}},
(A_i^D)_{i \in \mathbb{I}},
(B_i)_{i \in \mathbb{I}},
(C_i)_{i \in \mathbb{I}},
$ and $
(u_{i, -1})_{i \in \mathbb{I}}
$ as input.
The free variables are chosen, such that the objective function is minimized.

\subsubsection{Constraints}

As the optimal inputs are computed, the optimizer has to also take the constraints into account.
These are given as equations or inequations.
In this case there is one equation (\ref{formula:minlp.load}) and one inequation (\ref{formula:minlp.power}).

The equation (\ref{formula:minlp.load}) makes sure, that the combined power
of all power plants equals the power demand or ''load'' at each time instance $t$.
This is achieved by setting each constant $(l_t)_{t \in \mathbb{T}}$
equal to the sum of the power of all power plants at that time $t$.
The summands on the right-hand side are the multiplication of
$u_{i, t}$ and $p_{i, t}'$ and not simply $p_{i, t}'$,
because in this model the power output can be greater than 0
when the unit is not committet.
This is a modeling trick, in order to enable the modeling of the inequation (\ref{formula:minlp.power}).

The inequation (\ref{formula:minlp.power}) makes sure, that the power output $p_{i, t}'$
of each power plant $i$ at each time intance $t$ is within the limits of the power plant.
These limits are $P_{min, i}$ for the minimum and $P_{max, i}$ for the maximum
power output.
As mentioned earlier, the power output $p_{i, t}'$ never reaches 0,
if $P_{min, i}$ is greater than 0.
So after finding the minimum inputs $u$ and $p'$ for the MINLP problem,
the solution to the UCP is the actual power $p_{i, t} = u_{i, t} p_{i, t}'$.

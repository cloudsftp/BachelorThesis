\section{Method}

In the beginning, there will be an explanation of the most important concepts
that are used later in the main part.
These are smart grids and the unit commitment problem respectively,
as well as quantum computing (gate-based and quantum annealing).\\

\noindent In the main part, there will be two approaches:

One practical approach, where optimizations on real quantum computers are executed,
to test the feasibility of D-Wave's quantum annealer ''Advantage''.
As input to the unit commitment problem, real-world data is used, to get a rather realistic example problem.
Optimizations are considered practical if they apply to small towns or even small villages
with multiple producers and consumers of energy.

The other will be a theoretical approach, where the feasibility of future gate-based quantum computers
for practical optimization problems is discussed.

Both the practical and the theoretical performances of the quantum computers
are compared to the performance of classical computers.
The same unit commitment problems are run with the same input to get a good comparison.

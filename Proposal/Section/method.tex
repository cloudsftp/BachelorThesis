\section{Method}

In the beginning, there will be an explanation of the most important concepts
that are used later in the main part.
First of all smart grids and the motivation for those are explained.
Furthermore, the unit commitment problem and it's optimization implementation on classical computers
is explored.
After that, quantum computers and the optimization algorithms for the unit commitment problem are explained.
There are different architectures of quantum computers,
of which the gate-based quantum computers and quantum annealers are discussed. \\

\noindent In the main part, there will be two approaches:

One practical approach, where optimizations on real quantum computers are executed,
to test the feasibility of D-Wave's quantum annealer ''Advantage''.
As input to the unit commitment problem, real-world data is used, to get a rather realistic example problem.
Optimizations are considered practical if they apply to small towns or even small villages
with multiple producers and consumers of energy.

The other will be a theoretical approach, where the feasibility of future gate-based quantum computers
for practical optimization problems is discussed. \\

Results of classical computers and quantum computers are then compared
regarding their performance and quality.
To get a good comparison, the same unit commitment problems are run on classical computers with the same input.
